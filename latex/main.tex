\documentclass[10pt, reqno]{exam}
\usepackage[utf8]{inputenc}
\usepackage{amsthm}
\usepackage{libertine}
\usepackage[margin=0.25in, includefoot]{geometry}
\usepackage{amsmath,amssymb}
\usepackage{multicol}
\usepackage[shortlabels]{enumitem}
\usepackage{cancel}
\usepackage{listings}
\usepackage{tikz}
\usepackage{float}
\usepackage{mathtools}
\usepackage{wrapfig, lipsum}
\usepackage{chemfig}
\usepackage{chemmacros}
\usepackage{chemformula}
\usepackage{empheq}
\usepackage{caption}
\usepackage{subcaption}
\usepackage[normalem]{ulem}
\usepackage{graphicx}
\usepackage{bm}
\usepackage{accents}
\usepackage[T1]{fontenc}
\usepackage[font=small,labelfont=bf,tableposition=top]{caption}
\usepackage{siunitx}
\usepackage{minted}
\usepackage{etoolbox}
\usepackage{multirow}
\usepackage[version=4]{mhchem}
\usepackage{pgfplots}
\usepackage{svg}
\usepackage{upgreek}
\usepackage{longtable}
\usepackage{graphics}

\pgfplotsset{compat=1.18}

\AtBeginEnvironment{minted}{
    \fontsize{8}{10}\selectfont}
\newcommand{\fahrenheit}{^\circ{F}}
\newcommand*\widefbox[1]{\fbox{\hspace{2em}#1\hspace{2em}}}
\newcommand{\msout}[1]{
    \text{\sout{$#1$}}
}
\usetikzlibrary{shapes}
\DeclareSIUnit\year{y}
\sisetup{group-digits = integer, group-minimum-digits = 3, group-separator = {,}}
\newcommand{\class}{Dr. Shao's Research Group}
\newcommand{\examnum}{eSRIM Code Walkthrough}
\newcommand{\examdate}{Feburary 17\textsuperscript{th}, 2024}
\newcommand{\timelimit}{}
\newcommand{\laplace}{\mathcal{L}}

\begin{document}

\begingroup
\hbadness=10000
\noindent 
\begin{tabular*}{\textwidth}{l @{\extracolsep{\fill}} r @{\extracolsep{4pt}} l}
  \textbf{\class} & \textbf{Name:} & \textit{Nathaniel Thomas}\\ %Your name here instead, obviously 
  \textbf{\examnum}  && \\
  \textbf{\examdate} && \\
\end{tabular*}\\


\noindent\hrule width \textwidth height 2pt

\section{Notes on Syntax}

This section will quickly explain the most important syntax to be aware of in QuickBASIC. This language is unlike many other languages and requires some additional familiarization before you can fully understand how this program works. \par

\subsection{Comments}

All comments in QuickBASIC are indicated using a single quote, "'". Any text following this symbol is treated as a comment, and does not affect the functionality of the code. 

\subsection{Keywords}

There are several reserved "keywords" in QuickBASIC which carry special meaning to the QuickBASIC compiler. Table \ref{tbl:keywords} contains a largely comprehensive list of all of these keywords, as well as their purpose:

% Generated using ChatGPT 3.5, prompt: 
% I am tabulating all of the keywords in QuickBASIC, along with their type (Function, Operator, etc.), use ( varA AND varB, for example), and purpose. This is a template with the ABS keyword.

% \begin{table}[h]
%     \centering
%     \begin{tabular}{|c|c|c|c|c|}
%         \hline
%         Keyword & Type & Use  & Purpose  \\
%         \hline
%         ABS & Function & ABS(x), x is a SINGLE, DOUBLE, or INT & Calculate absolute value of a number \\
%         \hline
%     \end{tabular}
%     \caption{Comprehensive list of all QuickBASIC Keywords}
%     \label{tbl:keywords}
% \end{table}

% Please provide a complete version of this table for all  QuickBASIC keywords.

{
\footnotesize
\begin{longtable}{|c|c|c|c|}
    \caption{Comprehensive list of all QuickBASIC Keywords}
    \label{tbl:keywords} \\
    \hline
    Keyword & Type & Use & Purpose \\
    \hline
    ABS & Function & ABS(x) & Calculate absolute value of a number \\
    ACCESS & Statement & ACCESS mode, path & Set or return the access mode of a file \\
    AND & Operator & varA AND varB & Logical AND operation \\
    ANY & Operator & varA ANY varB & Bitwise OR operation \\
    APPEND & Statement & APPEND mode, path & Open a file for appending data \\
    AS & Keyword & DIM var AS type & Declare a variable with a specific data type \\
    ASC & Function & ASC(string) & Return ASCII code of the first character in a string \\
    ATN & Function & ATN(x) & Return arctangent of a number \\
    BASE & Statement & BASE value & Set or return the default base for numeric input and output \\
    BEEP & Statement & BEEP & Produce a beep sound \\
    BLOAD & Statement & BLOAD "filename", address & Load binary data from a file into memory \\
    BSAVE & Statement & BSAVE "filename", address, length & Save binary data from memory to a file \\
    BYVAL & Keyword & BYVAL parameter & Pass a parameter by value in a procedure \\
    CALL & Statement & CALL subroutine & Call a subroutine or function \\
    CALLS & Function & CALLS & Return number of subroutine or function calls \\
    CASE & Keyword & SELECT CASE expression & Define a case in a SELECT CASE structure \\
    CHAIN & Statement & CHAIN "filename" & Start execution of another program file \\
    CHDIR & Statement & CHDIR "path" & Change current directory \\
    CHR\$ & Function & CHR\$(code) & Return character associated with ASCII code \\
    CIRCLE & Statement & CIRCLE (x, y), radius & Draw a circle on the screen \\
    CLEAR & Statement & CLEAR & Clear all variable values \\
    CLNG & Function & CLNG(expression) & Convert expression to a long integer \\
    CLOSE & Statement & CLOSE [fileNumber] & Close an open file \\
    CLS & Statement & CLS & Clear the screen \\
    COLOR & Statement & COLOR foreground, background & Set text and background colors \\
    COMMON & Statement & COMMON [sharedVariable] & Declare shared variables \\
    CONST & Statement & CONST constantName = value & Define a constant \\
    COS & Function & COS(angle) & Return cosine of an angle \\
    CSRLIN & Function & CSRLIN & Return current cursor line position \\
    CVD & Function & CVD(string) & Convert string to a date value \\
    CVDMBF & Function & CVDMBF(string) & Convert string to a date value using the default date format \\
    CVI & Function & CVI(string) & Convert string to an integer value \\
    CVL & Function & CVL(string) & Convert string to a long integer value \\
    CVS & Function & CVS(string) & Convert string to a single-precision floating-point value \\
    DATA & Statement & DATA value1, value2, ... & Define data for READ statement \\
    DATE\$ & Function & DATE\$ & Return current date \\
    DECLARE & Statement & DECLARE subName [LIB "libraryName"] & Declare external subroutines or functions \\
    DEF FN & Statement & DEF FN functionName (parameters) = expression & Define a user-defined function \\
    DEF SEG & Statement & DEF SEG = segmentAddress & Set default segment for memory access \\
    DEFDBL & Statement & DEFDBL variableList & Declare double-precision variables by default \\
    DEFINT & Statement & DEFINT variableList & Declare integer variables by default \\
    DEFLNG & Statement & DEFLNG variableList & Declare long integer variables by default \\
    DEFSNG & Statement & DEFSNG variableList & Declare single-precision variables by default \\
    DEFSTR & Statement & DEFSTR variableList & Declare string variables by default \\
    DEFVAR & Statement & DEFVAR variableList & Declare variables with specific data types by default \\
    DELETE & Statement & DELETE & Delete a line in a program \\
    DIM & Statement & DIM array(dimensions) & Declare array variables \\
    DO & Statement & DO & Start a DO...LOOP loop \\
    DOUBLE & Type & DOUBLE & Double-precision floating-point data type \\
    DRAW & Statement & DRAW "action" & Draw graphics primitives \\
    DSKO\$ & Function & DSKO\$(drive) & Return description of disk \\
    DSKI\$ & Function & DSKI\$(drive) & Return information about disk \\
    ELSE & Keyword & IF condition THEN statement ELSE statement & Execute if condition is false \\
    ELSEIF & Keyword & IF condition THEN statement ELSEIF condition THEN statement & Execute if previous condition is false and current condition is true \\
    END & Statement & END & Terminate program execution \\
    END FUNCTION & Statement & END FUNCTION & End a function definition \\
    END IF & Statement & END IF & End an IF...THEN...ELSE block \\
    END SELECT & Statement & END SELECT & End a SELECT CASE block \\
    END SUB & Statement & END SUB & End a subroutine definition \\
    END TYPE & Statement & END TYPE & End a user-defined type definition \\
    END WHILE & Statement & END WHILE & End a WHILE...WEND loop \\
    ENVIRON\$ & Function & ENVIRON\$(name) & Return value of an environment variable \\
    EOF & Function & EOF(fileNumber) & Test for end of file \\
    ERASE & Statement & ERASE array & Delete array variables \\
    ERDEV & Function & ERDEV & Return error device number \\
    ERDEV\$ & Function & ERDEV\$ & Return error device name \\
    ERL & Function & ERL & Return line number of last error \\
    ERR & Function & ERR & Return error number \\
    ERROR\$ & Function & ERROR\$ & Return error message \\
    EXIT & Statement & EXIT DO & Exit a DO...LOOP loop \\
    EXP & Function & EXP(x) & Return exponential value of a number \\
    FIELD & Statement & FIELD [fileNumber], fieldNumber AS fieldDefinition & Define random-access file records \\
    FILES\$ & Function & FILES\$ & Return files in the current directory \\
    FILL & Statement & FILL & Fill graphics screen with a color \\
    FIX & Function & FIX(number) & Return integer part of a number \\
    FNEND & Statement & FNEND & End a user-defined function \\
    FOR & Statement & FOR counter = start TO end [STEP step] & Start a FOR...NEXT loop \\
    FORMAT\$ & Function & FORMAT\$(expression, format) & Format a numeric value \\
    FRAME & Statement & FRAME & Draw a frame around the graphics screen \\
    FREEFILE & Function & FREEFILE & Return a free file number \\
    FREQUENCY & Function & FREQUENCY & Return timer frequency \\
    FUNCTION & Statement & FUNCTION functionName (parameters) & Define a function \\
    GET & Statement & GET [fileNumber, ]recordNumber & Read a record from a random-access file \\
    GOSUB & Statement & GOSUB lineLabel & Call a subroutine \\
    GOTO & Statement & GOTO lineLabel & Unconditional jump to a line label \\
    HEX\$ & Function & HEX\$(number) & Return hexadecimal representation of a number \\
    IF & Statement & IF condition THEN statement & Execute if condition is true \\
    IMP & Operator & varA IMP varB & Logical implication \\
    INKEY\$ & Function & INKEY\$ & Check for a key press \\
    INP & Function & INP(port) & Read from an I/O port \\
    INPUT & Statement & INPUT prompt; variable & Prompt user for input \\
    INPUT\$ & Function & INPUT\$(prompt) & Prompt user for input and return string \\
    INPUT\# & Statement & INPUT\# fileNumber, variable & Read from an input file \\
    INSTR & Function & INSTR([start, ]string1, string2) & Return position of one string within another \\
    INT & Function & INT(number) & Return integer part of a number \\
    IRQ & Function & IRQ(number) & Return interrupt request status \\
    KILL & Statement & KILL "filename" & Delete a file \\
    LEFT\$ & Function & LEFT\$(string, count) & Return leftmost characters of a string \\
    LEN & Function & LEN(string) & Return length of a string \\
    LET & Statement & LET variable = expression & Assign a value to a variable \\
    LINE & Statement & LINE (x1, y1)-(x2, y2) & Draw a line on the screen \\
    LINE INPUT & Statement & LINE INPUT prompt; variable & Prompt user for input and read a line \\
    LINE INPUT\# & Statement & LINE INPUT\# fileNumber, variable & Read a line from an input file \\
    LOAD & Statement & LOAD "filename" & Load and run a program \\
    LOCATE & Statement & LOCATE row, column & Move cursor to specified position \\
    LOCK & Statement & LOCK [recordNumber] & Lock a record in a random-access file \\
    LOF & Function & LOF(fileNumber) & Return length of an open file \\
    LOG & Function & LOG(number) & Return natural logarithm of a number \\
    LONG & Type & LONG & Long integer data type \\
    LPOS & Function & LPOS & Return current cursor column position \\
    LPRINT & Statement & LPRINT expression & Print to the printer \\
    LSET & Statement & LSET variable = string & Assign a string to a variable \\
    LTRIM\$ & Function & LTRIM\$(string) & Return string with leading spaces removed \\
    MID\$ & Function & MID\$(string, start[, length]) & Return a substring of a string \\
    MKD\$ & Function & MKD\$(dateValue) & Convert date value to string \\
    MKDIR & Statement & MKDIR "path" & Create a directory \\
    MKI\$ & Function & MKI\$(integerValue) & Convert integer value to string \\
    MKL & Function & MKL(longValue) & Convert long integer value to string \\
    MKS\$ & Function & MKS\$(singleValue) & Convert single-precision value to string \\
    MOD & Operator & varA MOD varB & Modulus (remainder) operation \\
    NAME & Statement & NAME "oldName" AS "newName" & Rename a file or directory \\
    NEW & Statement & NEW & Clear memory and reset program execution \\
    NEXT & Statement & NEXT counter & Increment counter in a FOR...NEXT loop \\
    NOT & Operator & NOT expression & Logical negation \\
    OCT\$ & Function & OCT\$(number) & Return octal representation of a number \\
    ON & Statement & ON expression GOSUB lineLabel & Define a line label for GOSUB or GOTO \\
    OPEN & Statement & OPEN mode, fileNumber, "filename" & Open a file for input, output, or append \\
    OPTION & Statement & OPTION BASE number & Set the base index for array dimensions \\
    OR & Operator & varA OR varB & Logical OR operation \\
    OUT & Statement & OUT port, value & Write to an I/O port \\
    OUTPUT & Statement & OUTPUT & Switch output to the screen \\
    OVERLAY & Statement & OVERLAY "filename" & Overlay the current program with another program \\
    PAINT & Statement & PAINT (x, y), fillColor, borderColor & Fill an enclosed area with color \\
    PEN & Statement & PEN color & Set pen color for graphics drawing \\
    PLAY & Statement & PLAY "notes" & Play music notes \\
    POINT & Function & POINT(x, y) & Return color of a point on the screen \\
    POKE & Statement & POKE address, value & Write a value to a memory address \\
    POS & Function & POS & Return current cursor position \\
    PRINT & Statement & PRINT expression & Print to the screen \\
    PRINT USING & Statement & PRINT USING format; expression & Print using a specified format \\
    PSET & Statement & PSET (x, y), color & Set a pixel to a specified color \\
    PUT & Statement & PUT [fileNumber, ]recordNumber & Write a record to a random-access file \\
    RANDOMIZE & Statement & RANDOMIZE [seed] & Initialize random number generator \\
    READ & Statement & READ variableList & Read data from a DATA statement \\
    REDIM & Statement & REDIM array(dimensions) & Redimension an array \\
    REM & Statement & REM comment & Add a remark or comment to the code \\
    RESTORE & Statement & RESTORE [lineLabel] & Reset data pointer for READ statement \\
    RESUME & Statement & RESUME [lineLabel] & Resume execution after an error \\
    RETURN & Statement & RETURN [expression] & Return from a subroutine or function \\
    RIGHT\$ & Function & RIGHT\$(string, count) & Return rightmost characters of a string \\
    RMDIR & Statement & RMDIR "path" & Remove a directory \\
    RND & Function & RND & Return a random number \\
    RSET & Statement & RSET string = expression & Assign a string to a variable, right-justified \\
    RUN & Statement & RUN "filename" & Run another program \\
    SAVE & Statement & SAVE "filename" & Save the current program to a file \\
    SCREEN & Statement & SCREEN mode & Set video mode and screen resolution \\
    SEG & Function & SEG(variable) & Return segment part of a memory address \\
    SELECT & Statement & SELECT CASE expression & Start a SELECT CASE block \\
    SET & Statement & SET statement & Set environment attributes \\
    SETATTR & Statement & SETATTR "filename", attribute & Set file attributes \\
    SGN & Function & SGN(number) & Return sign of a number \\
    SIN & Function & SIN(angle) & Return sine of an angle \\
    SOUND & Statement & SOUND frequency, duration & Generate sound \\
    SPACE\$ & Function & SPACE\$(count) & Return a string of spaces \\
    SPC & Function & SPC(count) & Print a number of spaces \\
    SQR & Function & SQR(number) & Return square root of a number \\
    STATIC & Statement & STATIC variable & Declare a static variable \\
    STICK & Function & STICK(number) & Return joystick position \\
    STOP & Statement & STOP & Stop program execution \\
    STR\$ & Function & STR\$(number) & Convert number to string \\
    STRING\$ & Function & STRING\$(count, character) & Return a string of repeated characters \\
    SUB & Statement & SUB subName (parameters) & Define a subroutine \\
    SWAP & Statement & SWAP varA, varB & Exchange values of two variables \\
    SYSTEM & Statement & SYSTEM & Execute an operating system command \\
    TAB & Function & TAB(position) & Move cursor to a specified column \\
    TAN & Function & TAN(angle) & Return tangent of an angle \\
    THEN & Keyword & IF condition THEN statement & Execute if condition is true \\
    TIME\$ & Function & TIME\$ & Return current time \\
    TIMER & Function & TIMER & Return timer value \\
    TO & Keyword & FOR counter = start TO end & Specify range in a FOR...NEXT loop \\
    TRIM\$ & Function & TRIM\$(string) & Return string with leading and trailing spaces removed \\
    TYPE & Statement & TYPE typeName & Define a user-defined type \\
    UBOUND & Function & UBOUND(array, [dimension]) & Return upper bound of an array \\
    UCASE\$ & Function & UCASE\$(string) & Convert string to uppercase \\
    UNLOCK & Statement & UNLOCK [recordNumber] & Unlock a record in a random-access file \\
    UNTIL & Keyword & DO UNTIL condition & End a DO...LOOP loop until condition is true \\
    USING & Statement & PRINT USING format; expression & Print using a specified format \\
    VAL & Function & VAL(string) & Convert string to numeric value \\
    VIEW & Statement & VIEW PRINT [ON | OFF] & Control output to the screen \\
    WAIT & Statement & WAIT time & Pause program execution for specified time \\
    WEND & Statement & WHILE condition WEND & End a WHILE...WEND loop \\
    WHILE & Statement & WHILE condition & Start a WHILE...WEND loop \\
    WIDTH & Statement & WIDTH columns & Set width for screen output \\
    WINDOW & Statement & WINDOW (row1, col1)-(row2, col2) & Set window for screen output \\
    WRITE & Statement & WRITE fileNumber, record & Write to a sequential file \\
    XOR & Operator & varA XOR varB & Logical exclusive OR operation \\
    \hline
\end{longtable}
}

\pagebreak

\subsection{Reserved characters}

Table \ref{tbl:reserved characters} most of the reserved characters in QuickBASIC.

\begin{table}[h]
    \centering
    \caption{Comprehensive list of all QuickBASIC reserved characters}
    \label{tbl:reserved characters}
    \begin{tabular}{|c|c|c|c|}
    \hline
    Character & Type & Use & Purpose \\
    \hline
    ' & Special Character & ' This is a comment & Indicates a comment \\
    " & Special Character & "Hello, world!" & Denotes string literals \\
    : & Special Character & statement1 : statement2 & Separates multiple statements \\
    , & Special Character & FUNCTION(arg1, arg2) & Separates function arguments \\
    ; & Special Character & PRINT expr1; expr2 & Suppresses newline in PRINT statement \\
    . & Special Character & 3.14159 & Represents the decimal point in numbers \\
    + & Operator & varA + varB & Adds two numbers or concatenates strings \\
    - & Operator & varA - varB & Subtracts one number from another \\
    * & Operator & varA * varB & Multiplies two numbers \\
    / & Operator & varA / varB & Divides one number by another \\
    < & Operator & varA < varB & Checks if varA is less than varB \\
    > & Operator & varA > varB & Checks if varA is greater than varB \\
    = & Operator & varA = varB & Assigns a value to a variable \\
    ( & Special Character & (expr1 + expr2) & Indicates the start of an expression \\
    ) & Special Character & (expr1 + expr2) & Indicates the end of an expression \\
    $[$ & Special Character & array[index] & Indicates the start of an array index \\
    $]$ & Special Character & array[index] & Indicates the end of an array index \\
    $\{$ & Special Character & { statement1; statement2 } & Indicates the start of a code block \\
    $\}$ & Special Character & { statement1; statement2 } & Indicates the end of a code block \\
    ! & Special Character & !variable & Indicates a single-precision floating-point variable \\
    % & & & (some dialects) \\
    % & & & \\
    \hline
    \end{tabular}
\end{table}
\section{The Code}
\textbf{Note that any bolded text indicates an unknown value or an unanswered question.}
\subsection{Initialization}
\subsubsection{Function Declarations}

A number of functions and subroutines are declared at the start of the program. This is a feature of quickBASIC, where functions and subroutines must be declared first and then defined. The behavior of each of these functions and subroutines will be described in detail later on.

\begin{minted}[fontsize=\tiny,mathescape=false,linenos]{quickbasic.py:QuickBASICLexer -x}
    DECLARE SUB EMAGIC (X!)
    DECLARE SUB TMAGIC (MASS1!, Z1!, MASS2!, Z2!, INELAB!, P!)
    DECLARE SUB AMAGIC (THETAO!, ALPHAO!, THETA1RELATIVE!, THETA2RELATIVE!)
    DECLARE FUNCTION DF! (X!, COLUMBIAVK!, Z1!, Z2!, AU!)
    DECLARE FUNCTION F! (X!, COLUMBIAVK, Z1, Z2, AU)
    DECLARE SUB DSCMOTT (mott(), scr(), ElectronEnergy, SubstrateZ, Theta)
    DECLARE SUB IoniElecLoss (ZSUB, DENSITY, ElecE)
    DECLARE SUB BremsEloss (ZSUB, DENSITY, ElecE)
    DECLARE SUB IMAGE(Scan(),NewDeltaZ, NewDeltaX, NewDeltaY, NewTheta, NewAlpha, NewEP, Resolution,size)
\end{minted}
\subsubsection{Constants and Variables} 
A number of constants and variables are defined in this portion of the code. They are shown in Table \ref{tbl:constants_and_variables}
\pagebreak


{    
    \begin{longtable}{|c|c|c|c|c|}
        \caption{Table of constants/variables used in the eSRIM program}
        \label{tbl:constants_and_variables} \\
        \hline
        Name & Constant? & Data Type & Purpose & Unit \\
        \hline
        e\_range & Yes & INTEGER & Depth of interest for the simulation. & Angstrom \\
        e\_interval & Yes & INTEGER & Number of intervals for storing electron penetration data.* & Dimensionless \\
        e\_ave\_range & Yes & INTEGER & Unknown* & Unknown* \\
        e\_deltaLat & No & FLOAT & Used to calculate the length of an e-penetration interval. & Angstrom \\
        e\_deltaDep & No & FLOAT & Used to calculate the height of an e-penetration interval. & Angstrom \\
        e\_stopping & Yes & FLOAT & Stopping energy of electrons & keV \\
        ion\_stopping & Yes & FLOAT & Stopping energy of ions & keV \\
        ion\_Ed & Yes & FLOAT & Threshold displacement energy & keV \\
        Divisor & Yes & INTEGER & Number of angle intervals from 0-180 for Mott scattering & Dimensionless \\
        fly0 & Yes & INTEGER & Number of groupings for very low energy and high energy electrons & None \\
        fly1 & Yes & INTEGER & Number of groupings for standard energy-range electrons & None \\
        flyC & No & INTEGER & The current grouping number for electrons & None \\
        flyjudge & Yes & INTEGER & Used to track the critical energy for grouping-number switching. & None \\
        sizeScan & & & & \\
        Resolution & & & & \\
        scandelta & & & & \\
        EMAX & & & & \\
        switch & & & & \\
        \hline
        totalRE &  & DOUBLE & \textbf{Unknown} & \textbf{Unknown} \\
        ElecRE  &  &   &   &   \\
        EP  &   &   &   &   \\
        ElecREup    &   &   &   &   \\
        ElecREdown  &   &   &   &   \\
        L   &   &   &   &   \\
        Lselected   &   &   &   &   \\
        TotalCross  &   &   &   &   \\
        DiffCross  &   &   &   &   \\
        IoniEloss1  &   &   &   &   \\
        IoniEloss2  &   &   &   &   \\
        BEloss  &   &   &   &   \\
        E0  & No & SINGLE  & Stores radial stopping distance of a projectile from the z-axis & Angstrom \\
        E1  & No  & SINGLE  & Stores longitudinal depth of projectile & Angstrom  \\
        E2  &   & SINGLE  &   &   \\
        E3  &   & SINGLE  &   &   \\
        E4V     & SINGLE  &   &   &   \\
        \hline
        iontype & No & Integer & Stores the type of bombardment particle (1 is electron, 0 is ion.) & None \\
        ElecE0 & No & SINGLE & Stores electron energy & keV \\
        Elecmassp & Yes & SINGLE & The mass of an electron & amu \\
        ElecCP & No & INTEGER & The charge of an electron & e \\
        ePRbin & Yes & SINGLE & \textbf{Electron depth profile interval} & \si{\angstrom} \\
        massp & No & SINGLE & Mass of incident ion & amu \\
        CP & No & SINGLE & Charge of incident atom & e \\
        INELAB & No & SINGLE & Energy of incident atom & keV \\
        MASSSUB & No & SINGLE & Substrate atom mass & amu \\
        ZSUB & No & SINGLE & Substrate atom charge & e \\
        SUBDENSITY & No & SINGLE & Substrate atom density & \si{atoms/cm^3} $\rightarrow$ \si{atoms/\angstrom^3} \\
        SUBWINDOW & No & INTEGER & Plotting window area & \si{\angstrom} \\
        WINDOWX & Yes & Integer & Width of window plotting & pixels \\
        WINDOWY & Yes & Integer & Height of window plotting & pixels \\
        SIMULS & No & Integer & Number of particle bombardments to simulate & simulations \\
        PI & Yes & SINGLE & Constant for the number $\pi$ & None \\
        \hline
        MASS2 & No & SINGLE & Substrate atom mass, used in simulation & amu \\
        Z2 & No & SINGLE & Substrate atomic number, used in simulation & amu \\
        DENSITY & SINGLE & Substrate density, used in simulation & amu \\
        \hline
    \end{longtable}
}
\pagebreak
Here some of the values shown in Table \ref{tbl:constants_and_variables} are assigned. Note that these are variables that are not first declared.
\begin{minted}[fontsize=\tiny,mathescape=false,linenos,firstnumber=10]{quickbasic.py:QuickBASICLexer -x}
    e_range = 80000000 '7000000 'interested region depth in unit of Angsrom
    e_interval = 30 'interval number for the interest region
    e_ave_range = 1 'range to averageing neighboring data
    e_deltaLat = e_range / e_interval
    e_deltaDep = e_range / e_interval
    e_stopping = 1 '0.01 'electrons are assumed to stop at energies of 100 eV
    ion_stopping = 0.04 'ions are assumed to stop at energies of 20 eV
    ion_Ed = 0.04 'threshold displacement energy
    Divisor = 1000 'the number of angle intervals from 0 to 180 degrees for electron bombardment
    fly0 = 1000
    fly1 = 1000
    flyC = fly1
    flyjudge = 1 '20 '100
    sizeScan = 100
    Resolution = 10
    scandelta = 20
    EMAX = 10000
    switch = 1 'switch 0 is for explicit method, switch 1 is for middlepoint method, and switch 2 si for back (implicit) method
\end{minted}
\subsubsection{Arrays}

The "RANDOMIZE TIMER" command sets the seed for the random number generator to the current system time. Dividing by three divides the seed (the system time) by three. \textbf{It is not clear why this value is divided by 3.} \par

Following that, several different arrays are declared using the DIM command. For example, "DIM RN(flyC) AS DOUBLE" declares an array called "RN" with a size that is the same as the value of "flyC". The data type of the array entries are "DOUBLE", which is a 64-bit (generally) floating point number. Table \ref{tbl:arrays} lists the different arrays present in the program, as well as their purposes. Note that any lines missing the "AS <Datatype>" expression is automatically established as an array of SINGLEs, or single-precision floating point numbers, which are usually represented with 32 bits. \par

\begin{table}[h]
    \centering
    \caption{Table of arrays used in the eSRIM program}
    \footnotesize
    \begin{tabular}{|c|c|c|c|c|}
        \hline
        Name & Size & Data Type & Purpose & Unit \\
        \hline
        RN  & flyC & DOUBLE & \textbf{Unknown} & \textbf{Unknown} \\
        RL  & flyC + 1 & SINGLE & \textbf{Unknown} & \textbf{Unknown} \\
        RA  & flyC & DOUBLE & \textbf{Unknown} & \textbf{Unknown} \\
        RE  & flyC & DOUBLE & \textbf{Unknown} & \textbf{Unknown} \\
        mott & $96\times 5\times 6$ & SINGLE & Store parameters for Mott Scattering calculations & Various \\
        XP & 2000 & DOUBLE & Stores x-position of projectiles and target atoms & \si{\angstrom} \\
        YP & 2000 & DOUBLE & Stores y-position of projectiles and target atoms & \si{\angstrom} \\
        ZP & 2000 & DOUBLE & Stores z-position of projectiles and target atoms & \si{\angstrom} \\
        CP & 2000 & INTEGER & Stores charge number of projectile & e \\
        mp & 2000 & SINGLE & Stores mass of projective & amu \\
        THETAP & 2000 & SINGLE & Stores angle with respect to z-axis & radians \\
        ALPHAP & 2000 & SINGLE & Stores angle with respect to x-axis on yz-plane & radians \\
        EP & 2000 & DOUBLE & Stores energy of projectile & keV \\
        Ptype & 2000 & DOUBLE & Stores particle type (1 is electron, 0 is ion) & None \\
        ePR & 2000 & DOUBLE & Stores profile of projectile along the z axis & \textbf{None} \\
        THETA2 & 2000 & SINGLE & \textbf{Unknown} & radians \\
        ALPHA2 & 2000 & SINGLE & \textbf{Unknown} & radians \\
        Scan & sizeScan + 1 $\times$ sizeScan + 1 $\times$ 8 & SINGLE & For 2D plot of backscattered electrons & \textbf{None} \\
        e\_output & e\_interval$\times$ e\_interval$\times$ 4  &  SINGLE  & Stores location, energy loss, elastic scattering, and vacancy information for electrons & Various   \\
        \hline
    \end{tabular}
    \label{tbl:arrays} \\
\end{table}
{
\begin{minted}[fontsize=\tiny,mathescape=false,linenos,firstnumber=28]{quickbasic.py:QuickBASICLexer -x}
    RANDOMIZE TIMER / 3
    
    DIM RN(flyC) AS DOUBLE
    DIM RL(flyC + 1)
    DIM RA(flyC) AS DOUBLE
    DIM RE(flyC) AS DOUBLE
    DIM totalRE AS DOUBLE
    DIM ElecRE AS DOUBLE
    DIM EP AS DOUBLE
    DIM ElecREup AS DOUBLE
    DIM ElecREdown AS DOUBLE
    DIM L AS DOUBLE
    DIM Lselected AS DOUBLE
    DIM TotalCross AS DOUBLE
    DIM SHARED DiffCross AS DOUBLE
    DIM SHARED IoniEloss1 AS DOUBLE
    DIM SHARED IoniEloss2 AS DOUBLE
    DIM SHARED BEloss AS DOUBLE
    DIM E0 AS SINGLE
    DIM E1 AS SINGLE
    DIM E2 AS SINGLE
    DIM E3 AS SINGLE
    DIM E4V AS SINGLE
    DIM e_output(e_interval, e_interval, 4) AS SINGLE '1 for electron location, 2 for ionization energy loss, 3 for elastic scattering, 4 for vacancy production
    DIM Backup_e_output(e_interval, e_interval, 4) AS SINGLE
    DIM Ltotal AS DOUBLE
    
    Ltotal = 0.0
    
    
    DIM SHARED mott(96, 5, 6) ' Creating a matrix of size 96x5x6 to save parameters for Mott scattering. 96 is the number of elements (from Z=1 to Z=96) 
\end{minted}
\subsubsection{Reading in Mott Scattering Parameters}

This portion of the code reads the Mott Scattering Parameters in from literature, up to element 96. \par

\vspace{0.5 cm}

\begin{minted}[mathescape=false]{quickbasic.py:QuickBASICLexer -x}
OPEN "C:\Users\lshao.AUTH\Downloads\Mott.txt" FOR INPUT AS #1
\end{minted}

opens a file as the file handle "\#1" from the directory "C:\textbackslash Users\textbackslash lshao.AUTH\textbackslash Downloads\textbackslash Mott.txt". Note that this should be changed to the relevant directory on other machines. The file handle is closed once reading is done.


\begin{minted}[fontsize=\tiny,mathescape=false,linenos,firstnumber=54]{quickbasic.py:QuickBASICLexer -x}
    ROW = 1   ‘starting row number for data reading
    OPEN "C:\Users\lshao.AUTH\Downloads\Mott.txt" FOR INPUT AS #1 ‘Reading the input file for Mott scattering
    DO WHILE NOT EOF(1)   ‘Continue to read until the end of the input file
        INPUT #1, mott(ROW, 1, 1), mott(ROW, 1, 2), mott(ROW, 1, 3), mott(ROW, 1, 4), mott(ROW, 1, 5), mott(ROW, 1, 6)  ‘reading first row for Z=1, numbers are for mott(Z=1, 1, 1 to 6)
        INPUT #1, mott(ROW, 2, 1), mott(ROW, 2, 2), mott(ROW, 2, 3), mott(ROW, 2, 4), mott(ROW, 2, 5), mott(ROW, 2, 6) ‘reading first row for Z=1, numbers are for mott(Z=1, 2, 1 to 6)
        INPUT #1, mott(ROW, 3, 1), mott(ROW, 3, 2), mott(ROW, 3, 3), mott(ROW, 3, 4), mott(ROW, 3, 5), mott(ROW, 3, 6) ‘reading first row for Z=1, numbers are for mott(Z=1, 3, 1 to 6)
        INPUT #1, mott(ROW, 4, 1), mott(ROW, 4, 2), mott(ROW, 4, 3), mott(ROW, 4, 4), mott(ROW, 4, 5), mott(ROW, 4, 6) ‘reading first row for Z=1, numbers are for mott(Z=1, 4, 1 to 6)
        INPUT #1, mott(ROW, 5, 1), mott(ROW, 5, 2), mott(ROW, 5, 3), mott(ROW, 5, 4), mott(ROW, 5, 5), mott(ROW, 5, 6) ‘reading first row for Z=1, numbers are for mott(Z=1, 5, 1 to 6)
        ROW = ROW + 1   ‘after finishing reading ROW=1 for Z=1,  ROW changes to 2 for Z=2
    
    LOOP  ‘reading continues and restarts from line 75, with ROW increased by 1
    CLOSE #1 ' Close the file after reading the last input line

    PRINT "mott(26,1,1)="; mott(26, 1, 1) ‘print on the screen to check if the reading is accurate
    PRINT "mott(26,1,1)="; mott(26, 1, 1) ‘print on the screen to check if the reading is accurate
\end{minted}
\subsubsection{Reading in Electron Screening Parameters}

This portion of the code reads the Electron Screening Scattering Parameters in from literature, up to element 92. \par

\vspace{0.5 cm}

\begin{minted}[mathescape=false]{quickbasic.py:QuickBASICLexer -x}
    OPEN "C:\Users\lshao.AUTH\Downloads\screen factor.txt" FOR INPUT AS #2
\end{minted}

opens a file as the file handle "\#2" from the directory "C:\textbackslash Users\textbackslash lshao.AUTH\textbackslash Downloads\textbackslash screen factor.txt". Note that this should be changed to the relevant directory on other machines. The file handle is closed once the reading is done.

\begin{minted}[fontsize=\tiny,mathescape=false,linenos,firstnumber=69]{quickbasic.py:QuickBASICLexer -x}

    DIM SHARED scr(92, 6) ‘Creating a matrix of size 96x6 to save parameters for charge screening effect. 96 is the number of elements (from Z=1 to Z=96)
    ROW = 1 ‘starting row number for data reading
    OPEN "C:\Users\lshao.AUTH\Downloads\screen factor.txt" FOR INPUT AS #2 'Reading the input file for charge screening effect
    DO WHILE NOT EOF(2) ‘Continue to read until the end of the input file
        INPUT #2, scr(ROW, 1), scr(ROW, 2), scr(ROW, 3), scr(ROW, 4), scr(ROW, 5), scr(ROW, 6) ‘reading first row for Z=1, from scr(Z=1, 1) to scr(Z=1, to 6)
    
        ROW = ROW + 1  ‘after finishing reading ROW=1 for Z=1,  ROW changes to 2 for Z=2
    LOOP ‘reading continues and restarts from line 97, with ROW increased by 1
    CLOSE #2 ' Close the file after reading the last input line
    
    PRINT "scr(26,1)="; scr(26, 5) ) ‘print on the screen to check if the reading is accurate
    
\end{minted}
\subsubsection{Output file opening}

In this section of the code, a number of files are opened for output files. Table \ref{tbl:output files} shows all of the output files that are created for the program, as well as their purpose.

\begin{table}[h]
    \centering
    \caption{Output files produced by the program, as well as their file handles and purposes}
    \label{tbl:output files}
    \begin{tabular}{|c|c|c|}
        \hline
        File Name & File Handle & Purpose \\
        \hline
        PRANGEEd40.DAT  & 1 & \textbf{Unknown}  \\
        simulEd40.DAT   & 2 & \textbf{Unknown}  \\
        simulvimageEd40 & 3 & \textbf{Unknown}  \\
        3DINTLimageEd40.DAT & 4 & \textbf{Unknown}  \\
        eRangeimageEd40.dat & 5 & \textbf{Unknown}  \\
        ProjectedeRangeimageEd40    & 6 & \textbf{Unknown}  \\
        e-outimageEd40  & 7 & \textbf{Unknown}  \\
        Projected\_e\_rangeImageEd40.txt  & 9 & \textbf{Unknown}  \\
        20imageEd40.txt & 10 & \textbf{Unknown}  \\
        crossimageEd40.txt  & 11 & \textbf{Unknown}  \\
        imageEd40.txt   & 12 & \textbf{Unknown}  \\
        VancancyEd40.txt    & 13 & \textbf{Unknown}  \\
        dedx.txt    & 14 & \textbf{Unknown}  \\
        \hline
    \end{tabular}
\end{table}

Note that the directories used here are all inside of "C:\textbackslash Users\textbackslash lshao.AUTH\textbackslash ". To run this code on other machines, these lines must be changed to the correct directory.

\begin{minted}[fontsize=\tiny,mathescape=false,linenos,firstnumber=82]{quickbasic.py:QuickBASICLexer -x}
    ‘Lines 108 to 120 are for creating various output files to save data calculated. 
    OPEN "C:\Users\lshao.AUTH\Downloads\PRANGEEd40.DAT" FOR OUTPUT AS #1
    OPEN "C:\Users\lshao.AUTH\Downloads\simulEd40.DAT" FOR OUTPUT AS #2
    OPEN "C:\Users\lshao.AUTH\Downloads\simulvimageEd40" FOR OUTPUT AS #3
    OPEN "C:\Users\lshao.AUTH\Downloads\3DINTLimageEd40.DAT" FOR OUTPUT AS #4
    OPEN "C:\Users\lshao.AUTH\Downloads\eRangeimageEd40.dat" FOR OUTPUT AS #5
    OPEN "C:\Users\lshao.AUTH\Downloads\ProjectedeRangeimageEd40" FOR OUTPUT AS #6
    OPEN "C:\Users\lshao.AUTH\Downloads\e-outimageEd40.dat" FOR OUTPUT AS #7
    OPEN "C:\Users\lshao.AUTH\Downloads\Projected_e_rangeImageEd40.txt" FOR OUTPUT AS #9 '
    OPEN "C:\Users\lshao.AUTH\Downloads\20imageEd40.txt" FOR OUTPUT AS #10
    OPEN "C:\Users\lshao.AUTH\Downloads\crossimageEd40.txt" FOR OUTPUT AS #11
    OPEN "C:\Users\lshao.AUTH\Downloads\imageEd40.txt" FOR OUTPUT AS #12
    OPEN "C:\Users\lshao.AUTH\Downloads\VancancyEd40.txt" FOR OUTPUT AS #13
    OPEN "C:\Users\lshao.AUTH\Downloads\dedx.txt" FOR OUTPUT AS #14 '''''''''''''''''
\end{minted}
}
\subsubsection{User prompting}

At this point in the code, the user is prompted for a number of important parameters, such as whether this is an electron bombardment, the ion/electron energy, and substrate information, along with other important details. All of the variables assigned here are described in Table \ref{tbl:constants_and_variables}.

\begin{minted}[fontsize=\tiny,mathescape=false,linenos,firstnumber=96]{quickbasic.py:QuickBASICLexer -x}
    
    
    INPUT "Is this an electron bombardment (1=yes, 0=no)", iontype  ‘iontype decides whether it is an electron bombardment or ion bombardment. Default without typing = 0
    IF iontype = 1 THEN ‘Lines 126 to 132 are for the case of electron bombardment
        INPUT "Input electron energy(default=10 MeV)", ElecE0 ‘if it is electron bombardment, what is the electron energy? The energy is saved as ElecE0
        IF ElecE0 = 0 THEN ElecE0 = 10000  ‘if no typing of ElecE0, it becomes default (=1MeV=10000keV)
        Elecmassp = 5.45E-4 'mass of electron in the unit of amu
        ElecCP = -1 'charge of electron
        ePRbin = 1.5E6 'depth profile interval for electron in the unit of Angstroms
    
    ELSE      ‘if it is not electron bombardment, do nothing here, moves to the next line
    END IF  ‘the end of IF command 
    
    IF iontype = 0 THEN ‘lines 137 to 143 are for ion bombardment
        INPUT " Input incident atom mass?  (28)  ", massp 'MASSP--MASS OF INCIDENT ATOM
        IF massp = 0 THEN massp = 27.9  ‘if no typing, default is for silicon atomic mass
        INPUT " Input incident atom charge Z ?  (14)", CP 'ZP--CHARGE NUMBER OF INCIDENT ATOM
        IF CP = 0 THEN CP = 14 ‘if no typing, default is for silicon atomic number
        INPUT " Input incident atom energy (kev, default=50 keV)? ", INELAB 'IME-INCIDENT ATOM ENERGY (KEV) IN LAB COORDINATOR
        IF INELAB = 0 THEN INELAB = 50 ‘if not typing, default is 50 keV
    
    ELSE
    END IF
    
    INPUT " Input substrate atom mass? (55.85)", MASSSUB 'MASSSUB--MASS OF SUBSTRATE ATOM
    IF MASSSUB = 0 THEN MASSSUB = 55.85 ‘if no typing, default is for pure Fe substrate of atomic mass 55.85 amu
    INPUT " Input substrate atom charge Z? (26)", ZSUB 'ZSUB--ATOMIC NUMBER OF TARGET ATOM
    IF ZSUB = 0 THEN ZSUB = 26 ‘if not typing, default is for pure Fe of atomic number 26
    INPUT "Input substrate density (atoms/cc) (for C d=8.482E22/cc)", SUBDENSITY ‘input for substrate density in the unit of atoms per cc
    IF SUBDENSITY = 0 THEN SUBDENSITY = 8.482E+22 ‘if no input, atomic density is 8.482E22 per cc, for Fe
    SUBDENSITY = SUBDENSITY / 1E+24 ‘the atomic density changes to the unit of atoms per angstrom^3
    
    INPUT "SUBSTRATE WINDOW (A)", SUBWINDOW   ‘input for the depth region (in Angstrom) for plotting on the screen
    IF SUBWINDOW = 0 THEN SUBWINDOW = e_range ‘if no input, the plotting window width is the full range of interest. The value of e_range is defined already. 
    WINDOWX = 630   ‘Relevant to screen pixel for width, corresponding to depth in the longitudinal direction
    WINDOWY = 330   ‘relevant to screen pixel for height, corresponding to range in the lateral direction
    
    
    INPUT "HOW MANY SIMULATION YOU WANT(50)?", SIMULS  ‘input for how many particles to be simulated
    IF SIMULS = 0 THEN SIMULS = 1000 ‘if there is no input, 1000 particles are to be simulated
\end{minted}
\subsubsection{Additional Array Declarations}

More arrays are declared here, mostly for storing particle collision information        . The details are listed in Table \ref{tbl:arrays}. Several arrays are of size 2000. This is an arbitrary choice which is representative of the maximum collision recursion depth (number of generations of knock-on particles.). 

\begin{minted}[fontsize=\tiny,mathescape=false,linenos,firstnumber=136]{quickbasic.py:QuickBASICLexer -x}

    
    DIM XP(2000) AS DOUBLE 'saving X POSTION of ion/electron projectiles and target atoms, up to 2000 in a single bombardment
    DIM YP(2000) AS DOUBLE 'saving Y POSITION of ion/electron projectile or target atoms
    DIM ZP(2000) AS DOUBLE 'saving Z(DEPTH) POSITION  of ion/electron projectile or target atoms
    DIM CP(2000) 'saving CHARGE NUMBER of ion/electron projectile or target atoms
    DIM mp(2000) 'saving MASS of ion/electron projectile or target atoms
    DIM THETAP(2000) 'saving ANGLE with respect to Z AXIS
    DIM ALPHAP(2000) 'saving ANGLE WITH respect to X AXIS on the YZ PLANE)
    DIM EP(2000) AS DOUBLE 'saving ENERGY of ion/electron projectile or target atoms
    DIM Ptype(2000) 'saving the particle type, type 0 =ion, type 1=electron
    DIM ePR(2000) ‘saving profile of ion/electron projectiles along z axis (one dimension) 
    DIM THETA2(2000)
    DIM ALPHA2(2000)
    DIM SHARED Scan(sizeScan + 1, sizeScan + 1, 8) 'AS DOUBLE 'For 2D plot of backscattered electrons
\end{minted}
\subsubsection{Matrix zeroing}  

This portion of the code runs through the matrices initialized for storing particle information and zeros all of the elements in each array. Some other variables are set or zerod as well.

\begin{minted}[fontsize=\tiny,mathescape=false,linenos,firstnumber=151]{quickbasic.py:QuickBASICLexer -x}

    ‘lines 187 to 199 for setting initial values of matrix as zero
    FOR rum = 1 TO 2000
        XP(rum) = 0
        ZP(rum) = 0
        YP(rum) = 0
        CP(rum) = 0
        mp(rum) = 0
        ALPHAP(rum) = 0
        THETAP(rum) = 0
        EP(rum) = 0
        Ptype(rum) = 0
        ePR(rum) = 0
    NEXT rum
    
    
    SCREEN 9 ‘setting screen color
    VIEW (1, 1)-(WINDOWX, WINDOWY), 15, 15 'setting screen size and color for plotting
    
    PI = 3.1415926#   ‘setting value for PI
    
    DOSE = 0
    SPUTTER = 0    ‘setting initial value of atoms being sputtered as zero
    
    
    FOR i = 0 TO sizeScan
        FOR j = 0 TO sizeScan
            Scan(i, j, 8) = 0 'zero the initial value for backscattered electron count
        NEXT j
    NEXT i
\end{minted}
\subsection{Primary bombardment}

This portion of the code simulates actual projectile bombardments. The number of bombardments that are simulated is taken as an argument in the SIMUL variable.

\subsubsection{Bombardment Initialization}

Here, a number of initialization steps are taken to prepare for the bombardment simulation. First, the random number generator is initialized to the system time divded by three. This only happens once for every 200 simulations. \par

Then, depending on whether the bombardment projectile is an ion or an electron, a number of particle collision data arrays are initialized with the energy, mass, and charge (for ions only) specified by the user, as well as the particle type. All particles are started at x,y,z = 0. \par

\textbf{Finally, the Scan array is populated with some information. The exact purpose of this is not known.}

\begin{minted}[fontsize=\tiny,mathescape=false,linenos,firstnumber=181]{quickbasic.py:QuickBASICLexer -x}


    FOR SIMUL = 1 TO SIMULS    ‘starting the bombardment 
    
        IF SIMUL - 200 * INT(SIMUL / 200) = 0 THEN RANDOMIZE TIMER / 3  ‘change random number seed every 200 particles
    
    
        IF iontype = 1 THEN 'iontype=1 means electron irradiation, lines 225 to 236 are for initial projectile conditions
            XP(1) = 0 ‘projectile starting location is at the original point with X=0
            YP(1) = 0 ‘projectile starting location is at the original point with Y=0
            ZP(1) = 0 'projectile starting location is at the original point with Z=0
            CP(1) = ElecCP   ‘electron charge
            mp(1) = Elecmassp 'electron mass
            THETAP(1) = 0 ‘initial incident direction is along the z axis
            ALPHAP(1) = 0 ‘initial projected direction vector is zero degree from x-axis
            EP(1) = ElecE0 ‘initial electron energy in keV
            Ptype(1) = 1  ‘particle type is electron
            correctFactor = 1  
    
        ELSE
        END IF
    
        IF iontype = 0 THEN 'iontype=0 means ion irradiation
            XP(1) = 0 ‘projectile starting location is at the original point with X=0
            YP(1) = 0 ‘projectile starting location is at the original point with Y=0
            ZP(1) = 0  ‘projectile starting location is at the original point with Z=0
            CP(1) = CP ‘projectile charge, i.e. CP=14 for silicon 
            mp(1) = massp ‘projectile mass, i.e. massp=28 for silicon
            THETAP(1) = 0  ‘initial incident direction is along the z axis
            ALPHAP(1) = 0  ‘initial projected direction vector is zero degree from x-axis
            EP(1) = INELAB 'initial ion bombardment energy in keV
            Ptype(1) = 0 ‘particle type is ion
        ELSE
        END IF
        'scanning ===============================
    
        FOR i = 0 TO sizeScan 'scanning
            FOR j = 0 TO sizeScan 'scanning
                '   Scan(i, j, 1) = e_deltaLat * (i - sizeScan / 2) 'scanning
                '  Scan(i, j, 2) = e_deltaLat * (i - sizeScan / 2) 'scanning
                Scan(i, j, 1) = scandelta * (i - sizeScan / 2) 'scanning
                Scan(i, j, 2) = scandelta * (j - sizeScan / 2) 'scanning
    
                Scan(i, j, 3) = 0 'depth scanning
                Scan(i, j, 4) = 0 'theta scanning
                Scan(i, j, 5) = 0 'alpha scanning
                Scan(i, j, 6) = ElecE0 'E scanning
                Scan(i, j, 7) = 0
    
            NEXT j 'scanning
        NEXT i 'scanning
        'scanning ======================
\end{minted}
\subsubsection{Bombardment bounds checking}

At this point, several variables are assigned. The "rum" variable is particularly important, as it is used for tracking recursive bombardment. When the initial particle is simulated, it has a "rum" value of 1. But when that particle hits another particle and transfer enough energy to cause it to begin moving like a projectile, it is given a value of "rum + 1", or a "rum" value of 2. This variable is used to track how deeply the particle collisions cascade. Due to the nature of the program, the maximum cascade depth is 2000, which was mentioned earlier on. \par

The substrate atomic mass, charge, and density are also assigned to simulation variables. \textbf{The reason why the original variables are not used is not clear.} \par

Some bounds checking statements are then run to ensure that the particle has not done any of the follwing:

\begin{enumerate}
    \item Flown out of the window: If the Z-position of the particle is greater than the size of the plotting window (SUBWINDOW), then the particle no longer needs to be simulated. The "rum" variable is decremented, as the current cascade depth is terminated, and the simulation continues simulating the "parent" (the particle which caused the current one to cascade).
    \item Scattered off the surface: If the Z-position of the particle is less than 0, then the particle bounced off the surface of the substrate. This is considered a back-scattered particle, which no longer needs to be simulated. A counter tracking the number of backscattered particles (SPUTTER) is incremented to account for the particle, and the simulation continues to the parent particle.
\end{enumerate}

\begin{minted}[fontsize=\tiny,mathescape=false,linenos,firstnumber=233]{quickbasic.py:QuickBASICLexer -x}
    
        rum = 1        ‘the first simulation of a new bombardment event 
        MASS2 = MASSSUB  ‘substrate atomic mass
        Z2 = ZSUB  ‘substrate atomic number (charge)
        DENSITY = SUBDENSITY ‘substrate density
    
        100
        IF rum = 0 THEN GOTO 600 ' If all collisions caused by one bombardment are finished,  start a new bombardment by jumping to 600
    
        IF ZP(rum) >= SUBWINDOW THEN 
            rum = rum – 1 ‘if electron/ion fly out of the window, stop the simulation and point to the next collision saved but not finished. 
            GOTO 100 ‘jumping to the saved collisions not yet finished. If one ion bombardment creates 900 displacements and if rum= #900 finished the collision, the pointer moves to #899 to finish the rest of collisions. #0 after rum=rum-1 means all collisions have finished. 
        ELSE
        END IF
        IF ZP(rum) < 0 THEN   ‘for the case that the particle is backscattered
            IF Ptype(rum) = 0 THEN SPUTTER = SPUTTER + 1  ‘if backscattered particle is ion, sputtering number increases by 1
            rum = rum – 1 ‘with backscattering or sputtering, no need to continue the collision, pointer moves to the next saved collision not yet finished. 
            GOTO 100
    
    
        ELSE
        END IF
\end{minted}
\subsubsection{Grouping number changes}

The code uses an optimization technique where several groups of "flying distances" (Essentailly, the random distance a projectile travels between atoms) are processed together in a group with a size of the variable "flyC". In the case of electrons, at low energies, this technique may slightly overestimate the amount of energy lost in a bombardment, resulting in a negative electron energy. This barrier is set to the value of the variable "flyjudge" multiplied by the value of the electron stopping energy (e\_stopping). To avoid this situation, the number of groupings are changed to a value (fly0) to allow for a more accurate simulation. \par

A different problem may arise for high energy electrons, where if an electron energy is beyond a certain value (EMAX), it may travel too far in a single flying distance grouping. \textbf{The exact reason why this is a problem is not known.} Because of this, the number of flying distances grouped is changed to a value to avoid this (fly0). \par

In all other cases, the number of groupings is set to fly1. \par

\textbf{There are a number of issues with this portion of the code however. Contrary to the comments, the value of fly0 and fly1 are equal (1000) and are not changed at any point in the code. Also, the value of flyjudge is also not changed, and is set to 1 at the start of the code. The result is that the value of flyC is always 1000 regardless of the energy of the projectile.}

\begin{minted}[fontsize=\tiny,mathescape=false,linenos,firstnumber=255]{quickbasic.py:QuickBASICLexer -x}

    
        IF Ptype(rum) = 1 AND EP(rum) < flyjudge * e_stopping THEN flyC = fly0   ‘if the projectile is an electron and if the electron energy is smaller than a value, the number of flying distances to be combined for evaluation is set to be fly0.  The critical value is flyjudge X e_stopping.  Flyjudge is a number defined in the input. e_stopping is the value simulations stop when electron energy is reduced to this value.  This command is used to avoid the situation that final electron energy become negative when the flying distance combination overestimates the energy loss. 
        IF Ptype(rum) = 1 AND EP(rum) >= flyjudge * e_stopping THEN flyC = fly1 ‘if the projectile is an electron and if electron energy is above flyjudge X e_stopping, the number of flying distances to be combined is fly1. This is to allow flying distance combination if electron energy is not too low.
        IF Ptype(rum) = 1 AND EP(rum) > EMAX THEN flyC = fly0  “if the energy of the electron is above EMAX, the number of combined free flying distance is fly0. This is used to avoid the situation that flying distance combination become too long as very high energy since a single free flying distance is large at high energy. 
\end{minted}
\subsubsection{Stopped particle information}

If a particle's energy is reduced to less than the threshold energies (e\_stopping for electrons, ion\_stopping for ions), then the particle's information is stored and written to the output data array (e\_output). \par

The radial stopping distance from the z-axis is calculated using Equation \ref{eq:radial distance}:

\begin{equation}
    z &= \sqrt{x^2 + y^2} 
    \label{eq:radial distance}
\end{equation}

\begin{minted}[fontsize=\tiny,mathescape=false,linenos,firstnumber=260]{quickbasic.py:QuickBASICLexer -x}

    
        ''    IF mp(rum) = MASS2 THEN PRINT #4, USING "##.###^^^^^ ##.###^^^^^ ##.###^^^^^"; XP(rum); YP(rum); ZP(rum) 'modified 01/12/01
        ''    I(INT(ZP(rum))) = I(INT(ZP(rum))) + 1
        ''   IF mp(rum) = massp THEN PR(INT(ZP(rum))) = PR(INT(ZP(rum))) + 1
        '       IF Ptype(rum) = 1 THEN PRINT #5, USING "##.###^^^^^ ##.###^^^^^ ##.###^^^^^"; XP(rum); YP(rum); ZP(rum)
        '       IF Ptype(rum) = 1 THEN ePR(1 + INT(ZP(rum) / ePRbin)) = ePR(1 + INT(ZP(rum) / ePRbin)) + 1
        IF Ptype(rum) = 1 AND EP(rum) < e_stopping THEN    ‘saving information if electrons stop. e_stopping is the threshold energy
    
            E0 = (XP(rum) ^ 2 + YP(rum) ^ 2) ^ 0.5 ‘radial distance from z axis
            E1 = ZP(rum)   ‘longitudinal depth of electrons when stopping
            IF E0 < SUBWINDOW AND E1 > 0 AND E1 < SUBWINDOW THEN ‘if electrons are in the valid region, not sputtered, and not beyond the window
                e_output(INT(1 + E0 / e_deltaLat), INT(1 + E1 / e_deltaDep), 1) = e_output(INT(1 + E0 / e_deltaLat), INT(1 + E1 / e_deltaDep), 1) + 1  ‘adding one to the saved counts at the 2-D position matrix of lateral distance and longitudinal depth. Both distances are divided by lateral direction interval (e_deltaLat) and longitudinal direction interval (e_deltaDep). Both intervals are predefined. INT() is to obtain the integral portion of a real number. 
            ELSE
            END IF
            rum = rum – 1   ‘after the stopped electron finishes the position counting, move to the previously saved not-yet-finished ion/atom, by pointing the index number to rum-1. 
            GOTO 100 ‘go back to start the next particle (other particles not-yet-finished, but produced from the same bombardment)
    
    
        ELSE
        END IF
    
        IF Ptype(rum) = 0 AND EP(rum) < ion_stopping THEN    ‘check if the new particle is an atom and the energy is below the threshold value
    
            E0 = (XP(rum) ^ 2 + YP(rum) ^ 2) ^ 0.5   ‘if yes, calculate the radial distance from z axis
            E1 = ZP(rum)  ‘longitudinal depth of electrons at the stopping position
            IF E0 < SUBWINDOW AND E1 > 0 AND E1 < SUBWINDOW THEN ‘judge if the ion stoops between two boundaries, surface and the maxim depth
                e_output(INT(1 + E0 / e_deltaLat), INT(1 + E1 / e_deltaDep), 4) = e_output(INT(1 + E0 / e_deltaLat), INT(1 + E1 / e_deltaDep), 4) + 1  ‘if yes, saving one count to 2-D position matrix (lateral position, longitudinal position). Both positions are divided by an distance interval. e_deltaLat is for the lateral and e_deltaDep is for the longitudinal 
            ELSE
            END IF
            rum = rum – 1  ‘after the stopped atom finishes the position counting, move to the previously saved not-yet-finished collision, by pointing the index number to rum-1. 
            GOTO 100 ‘go back to start the next particle (other particles not-yet-finished, but produced from the same bombardment)
    
    
        ELSE
        END IF
\end{minted}
\begin{minted}[fontsize=\tiny,mathescape=false,linenos,firstnumber=296]{quickbasic.py:QuickBASICLexer -x}   

    
        '================== BEGIN A NEW COLLISION
    
        IF Ptype(rum) = 0 THEN ‘check if the new collision is for an atom, instead of an electron
            KL = 1.212 * CP(rum) ^ (7 / 6) * Z2 / (CP(rum) ^ (2 / 3) + Z2 ^ (2 / 3)) ^ (3 / 2) / (mp(rum) ^ .5) ‘parameter for electronic stopping
            PL = .5  ‘another parameter for electronic stopping
            SE = KL * (EP(rum) * 1000) ^ PL 'electronic stopping at energy EP(), in the unit of eV
            L = DENSITY ^ (-1 / 3) ‘average atomic distance of the target, in the unit of Angstrom
            P = (RND / (PI * DENSITY ^ (2 / 3))) ^ .5  ‘using random number RND to select a collision parameter
    
    
            '================== COLLISON PARAMETER
    
            EP(rum) = EP(rum) - 1.59 * L * DENSITY * SE / 1000 ‘get an updated energy after considering electron energy loss. L is the average atomic distance selected as the step length. DENSITY is the atomic density in the unit of atoms per angstrom  ^3. SE is the electronic stopping power. 1/1000 is to convert the energy loss from eV to keV
    
            IF EP(rum) < ion_stopping THEN ‘check again if adding electronic stopping leads to  ion stopping. Ion_stopping is the energy criteria to stop. The stopping criteria is not necessary to be threshold displacement energy
                rum = rum – 1 ‘after the stopped atom finishes the position counting, move to the previously saved not-yet-finished collision, by pointing the index number to rum-1. 
    
                GOTO 100 ‘go back to start the next particle (other particles not-yet-finished, but produced from the same bombardment)
            ELSE
            END IF
            CALL TMAGIC(mp(rum), CP(rum), MASS2, Z2, EP(rum), P)  
            'TO get recoil energy RE, deviation angle from the incident direction for the projectile THETA1RELATIVE, , recoil direction of target atom with respect to the incident direction THETA2RELATIVE
            CALL AMAGIC(THETAP(rum), ALPHAP(rum), THETA1RELATIVE, THETA2RELATIVE)  ‘obtain the new directions with respect to the original xyz coordinate for the projectile and target atom (THETAP is the angle away from the z axis, ALPHAP is the angle away from the x axis on YZ plane); THETAP(RUM), ALPHA(RUM) are angles before collision; THETA1 and THETA2 are new angles relative to the direction before the collision
    
            '=====TEMPERARY SAVE INFORMATION.  Using matrix at index rum+1 to save everything about the target. Note the information about project is save with index rum
    
            ZP(rum + 1) = ZP(rum) + L * COS(THETAP(rum))   ‘assign new depth to target
            XP(rum + 1) = XP(rum) + L * SIN(THETAP(rum)) * COS(ALPHAP(rum)) ‘assign new X to target 
            YP(rum + 1) = YP(rum) + L * SIN(THETAP(rum)) * SIN(ALPHAP(rum)) ‘assign new Y to target
            CP(rum + 1) = Z2 'CP(rum + 1) = CP(rum)   ‘transfer target charge information
            mp(rum + 1) = MASS2 'mp(RUM + 1) = mp(RUM)  ‘transfer target mass information
            THETAP(rum + 1) = THETA2   ‘transfer angle information of Recoiled target. Note “2” for target. THETA2 is a shared parameter obtained from subroutine AMAGIC 
            ALPHAP(rum + 1) = ALPHA2 'transfer angle information of  Recoiled target. Note “2” for target. ALPHA2 is a shared parameter obtained from subroutine AAMAGIC
            EP(rum + 1) = RE  ‘transfer recoil energy to target atom energy. RE is a shared parameter obtained from subroutine TMAGIC
    
    
            WINDOW SCREEN(0, 0)-(WINDOWX, WINDOWY)  ‘define/draw a window
    
            LOCATE 4, 47 ‘locate position for words below
            PRINT " SIMULATION:"; SIMUL; "("; SIMULS; ")" 'provide updates on how many finished out of the total
    
    
            IF mp(rum) = massp THEN LINE (INT(ZP(rum) / SUBWINDOW * WINDOWX), INT((0.5 + XP(rum) / SUBWINDOW) * WINDOWY))-(INT(ZP(rum + 1) / SUBWINDOW * WINDOWX), INT((0.5 + XP(rum + 1) / SUBWINDOW) * WINDOWY)), 4 'draw a short line from previous position to the current position. Note all positions are normalized by the window size and then change to correct pixel number. The drawing is for projectile ion
            IF mp(rum) = MASSSUB THEN LINE (INT(ZP(rum) / SUBWINDOW * WINDOWX), INT((0.5 + XP(rum) / SUBWINDOW) * WINDOWY))-(INT(ZP(rum + 1) / SUBWINDOW * WINDOWX), INT((0.5 + XP(rum + 1) / SUBWINDOW) * WINDOWY)), 1 'similar to the above but the drawing is for target atoms with a different color
    
    
    
            IF RE <= ion_Ed THEN ‘if target atom energy is too low. No need to save information for target since there is no displacement. Saved information in rum+1 for the target is “selectively” transferred back to rum for the projectile.  
                ZP(rum) = ZP(rum + 1)  ‘return new Z back to projectile as an update
                XP(rum) = XP(rum + 1) ‘return new X back to projectile as an update
                YP(rum) = YP(rum + 1)  ‘return new Y back to projectile as an update
                mp(rum) = mp(rum)  ‘keep the original projectile mass
                CP(rum) = CP(rum) ‘keep the original projectile charge
                THETAP(rum) = THETA1  ‘update on the new direction for projectile, obtained from AMAGIC
                ALPHAP(rum) = ALPHA1 ‘update on the new direction for projectile, obtained from AMAGIC
                EP(rum) = EP(rum) - RE ‘update energy for projectile, considering energy transfer to the target. RE obtained from TMAIG
                GOTO 100    ‘return for the next step with new position and new energy. Note the index is kept at rum. It means the pointer stays on the same projectile.  All previous saved information on rum+1 are not used if target atom does not become a displacement
            ELSE
            END IF
    
            '====IF RE>0.02  A target displacement is created. The new atom needs to be assigned with rum+1.  From lines 411 to 420, the information transfer for rum+1 already happened. So, only the projectile needs to be updated for rum
    
            XP(rum) = XP(rum + 1)    ‘projectile has the same X as the target
            YP(rum) = YP(rum + 1) 'YP(RUM) = YP(RUM + 1) ‘projectile has the same Y as the target
            ZP(rum) = ZP(rum + 1) 'ZP(RUM) = ZP(RUM + 1) ‘projectile has the same Z as the target
            mp(rum) = mp(rum) 'projectile keeps its original mass
            CP(rum) = CP(rum) 'projectile keeps its original charge
            THETAP(rum) = THETA1 'projectile has its updated angle, obtained from AMAGIC. “1” is for projectile. TEHTA1 is shared from AMAGIC
            ALPHAP(rum) = ALPHA1 'projectile has its updated angle, obtained from AMAGIC. “1” is for projectile. ALPHA1 is shared from AMAGIC
            EP(rum) = EP(rum) - RE 'new energy considering the recoil energy loss
    
            IF ZP(rum + 1) > 0 AND ZP(rum + 1) < SUBWINDOW THEN  ‘Since one displacement occurs, vacancy information needs to be saved, if the displacement position is within two boundaries: surface and backside.
                '            V(INT(ZP(rum + 1))) = V(INT(ZP(rum + 1))) + 1  ‘optional for vacancy counting 
                E0 = (XP(rum) ^ 2 + YP(rum) ^ 2) ^ 0.5   ‘distance away from z axis
                E1 = ZP(rum)  ‘depth along z axis
                e_output(INT(1 + E0 / e_deltaLat), INT(1 + E1 / e_deltaDep), 4) = e_output(INT(1 + E0 / e_deltaLat), INT(1 + E1 / e_deltaDep), 4) + 1  ‘adding one vacancy into output matrix for e_output (Lateral, Longitudinal, 4). “4” is for vacancy. e_deltaLat and e_deltaDep are distance interval. INT is for taking integral number. 
    
            ELSE
            END IF
            rum = rum + 1   ‘since a new energetic atom is produced and taking the index rum+1, the calculation pointer is re-appointed to rum+1
            GOTO 100   ‘restart a new calculation with pointer at rum+1. If a displacement is created. The newly displaced atom will finish the simulation. Then the pointer moves back to the projectile to continue
        ELSE
        END IF
    
        'Below is for electron irradiation
    
        IF Ptype(rum) = 1 THEN   ‘”1” means the particle is electron
    
            FOR trial2 = 1 TO flyC   ‘create a chain of random number for free flying distance within the group.  flyC is the number of free flying distances in the group
                '    RANDOMIZE TIMER / 3   
                RL(trial2) = RND    ‘Assigned random number to random number matrix with ID of trial2. Trial2 starts from 1, increase till flyC. 
            NEXT trial2     ‘repeat until all numbers are assigned
    
            FOR trial3 = 1 TO flyC – 1   ‘first step to rank random number from low to high. Here is to pick from the original order.
                FOR trial4 = trial3 + 1 TO flyC  ‘compare with the rest after the one being picked 
                    IF RL(trial3) > RL(trial4) THEN  ‘judge whether there is random number behind is smaller than the picked one
                        temp = RL(trial3)  ‘assign a tempera saving for the originally picked number
                        RL(trial3) = RL(trial4)  ‘assigned the smaller random number to the originally picked one
                        RL(trial4) = temp  ‘transfer the save value of the original random to the one being swapped
                    END IF
                NEXT trial4   ‘finish everyone behind trial3
            NEXT trial3  ‘finish all trial3 from 1 to fly-1
    
            
            TotalCross = 0      ‘preparing to calculate integrated cross section 
            FOR ANGLE1 = 1 TO (Divisor - 1) ‘dividing 180 degrees by Divisor, performing integration
                CALL DSCMOTT(mott(), scr(), EP(rum) * correctFactor, ZSUB, PI / (1 - 10) * (1 - 10 ^ (ANGLE1 / Divisor)))    ‘obtaining differential cross section at specific angle. The angle reading is not uniform. It has higher density close to 0.
                IF ANGLE1 = 1 THEN TotalCross = TotalCross + DiffCross * 2 * PI * SIN(PI / (1 - 10) * (1 - 10 ^ (1 / Divisor))) * ((PI / 2 / (1 - 10) * (1 - 10 ^ (2 / Divisor))) + (PI / 2 / (1 - 10) * (1 - 10 ^ (1 / Divisor))))  ‘integration concerning the first interval which starts from angle=0
                IF ANGLE1 > 1 AND ANGLE1 < (Divisor - 1) THEN TotalCross = TotalCross + DiffCross * 2 * PI * SIN(PI / (1 - 10) * (1 - 10 ^ (ANGLE1 / Divisor))) * ((PI / 2 / (1 - 10) * (1 - 10 ^ ((ANGLE1 + 1) / Divisor))) - (PI / 2 / (1 - 10) * (1 - 10 ^ ((ANGLE1 - 1) / Divisor))))  ‘integration for angle intervals excluding two boundary points
                IF ANGLE1 = (Divisor - 1) THEN TotalCross = TotalCross + DiffCross * 2 * PI * SIN(PI / (1 - 10) * (1 - 10 ^ ((Divisor - 1) / Divisor))) * (PI - (PI / 2 / (1 - 10) * (1 - 10 ^ ((Divisor - 1) / Divisor))) - (PI / 2 / (1 - 10) * (1 - 10 ^ ((Divisor - 2) / Divisor))))  ‘integration concerning the last angle point concerning the boundary at 180 degree
                '    P1 = DiffCross
                '   P2 = TotalCross
                '  PRINT #11, USING "##.###^^^^^ ##.###^^^^^ ##.###^^^^^"; 180 / PI * PI / (1 - 10) * (1 - 10 ^ (ANGLE1 / Divisor)); P1; P2 'optional for saving differential cross section and integrated cross section as a function of angle, for checking 
            NEXT ANGLE1  ‘finish from 0 to 180
    
    
    
            Lselected = 0    ‘prepare for flying distance assignment
            FOR trial1 = 1 TO flyC ‘starting flying distance assignment with the group. flyC is the group size
    
                ForLselect = RND    ‘assign random number to ForLselect
                IF ForLselect = 0 THEN ForLselect = 1E-10  ‘if random number is zero, change to a very small number
                RN(trial1) = -LOG(ForLselect) / (SUBDENSITY * 1E24 * TotalCross) + (SUBDENSITY * 1E24) ^ (-1 / 3)
                '   Random number is converted to free flying distance using total cross section obtained from integration (lines 521 to 541)
                Lselected = Lselected + RN(trial1)  ‘adding each free flying distance to get the total length of the whole group
            NEXT trial1  ‘repeat and go through all free flying distance in the group
    
    
            ForThetaSelect0 = 0  ‘prepare to identify scattering angle. ThetaSelect0 is the integrated cross section. It is zero before the integration starts 
    
            trial5 = 1     ‘pointer needs to be updated
    
            RL(flyC + 1) = 12345.   ‘assignment of a “ridiculous angle” to the last scattering angle in the group
            FOR ANGLE2 = 1 TO (Divisor - 1)   ‘go through each angle point from 0 to 180
    
                CALL DSCMOTT(mott(), scr(), EP(rum) * correctFactor, ZSUB, PI / (1 - 10) * (1 - 10 ^ (ANGLE2 / Divisor)))  ‘obtain Mott differential cross section as a specific angle
    
                IF ANGLE2 = 1 THEN ForThetaSelect1 = ForThetaSelect0 + DiffCross / TotalCross * 2 * PI * SIN(PI / (1 - 10) * (1 - 10 ^ (1 / Divisor))) * ((PI / 2 / (1 - 10) * (1 - 10 ^ (2 / Divisor))) + (PI / 2 / (1 - 10) * (1 - 10 ^ (1 / Divisor))))    ‘Integration concerning first point at angle=0 needs to be specially treated. 
                IF ANGLE2 > 1 AND ANGLE2 < (Divisor - 1) THEN ForThetaSelect1 = ForThetaSelect0 + DiffCross / TotalCross * 2 * PI * SIN(PI / (1 - 10) * (1 - 10 ^ (ANGLE2 / Divisor))) * ((PI / 2 / (1 - 10) * (1 - 10 ^ ((ANGLE2 + 1) / Divisor))) - (PI / 2 / (1 - 10) * (1 - 10 ^ ((ANGLE2 - 1) / Divisor)))) ‘integration for the middle points without two angle boundaries
                IF ANGLE2 = (Divisor - 1) THEN ForThetaSelect1 = ForThetaSelect0 + DiffCross / TotalCross * 2 * PI * SIN(PI / (1 - 10) * (1 - 10 ^ ((Divisor - 1) / Divisor))) * (PI - (PI / 2 / (1 - 10) * (1 - 10 ^ ((Divisor - 1) / Divisor))) - (PI / 2 / (1 - 10) * (1 - 10 ^ ((Divisor - 2) / Divisor))))  ‘integration concerning the last boundary needs to be specially treated
    
                150
                IF ForThetaSelect1 > RL(trial5) THEN  ‘if the integration cross section is larger than the first random number, do the following
                    IF ANGLE2 > 1 AND ANGLE2 < Divisor - 1 THEN RA(trial5) = PI / 2 / (1 - 10) * (1 - 10 ^ ((ANGLE2 + 1) / Divisor)) + PI / 2 / (1 - 10) * (1 - 10 ^ (ANGLE2 / Divisor)) - (ForThetaSelect1 - RL(trial5)) * (PI / 2 / (1 - 10) * (1 - 10 ^ ((ANGLE2 + 1) / Divisor)) - PI / 2 / (1 - 10) * (1 - 10 ^ ((ANGLE2 - 1) / Divisor))) / (ForThetaSelect1 - ForThetaSelect0)   ‘if it happens for the middle point,  corresponding angle is read from the proportionality, judged by the distance from the right side boundary point.  RA is the scattering angle selected.
                    IF ANGLE2 = 1 THEN RA(trial5) = PI / 2 / (1 - 10) * (1 - 10 ^ (2 / Divisor)) + PI / 2 / (1 - 10) * (1 - 10 ^ (1 / Divisor)) - (ForThetaSelect1 - RL(trial5)) * (PI / 2 / (1 - 10) * (1 - 10 ^ (2 / Divisor)) + PI / 2 / (1 - 10) * (1 - 10 ^ (1 / Divisor))) / (ForThetaSelect1 - ForThetaSelect0)  ‘if it happens to the first angle interval, special treatment needs since interval width differs from middle points. RA is the scattering angle selected
                    IF ANGLE2 = Divisor - 1 THEN RA(trial5) = PI - (ForThetaSelect1 - RL(trial5)) * (PI - PI / 2 / (1 - 10) * (1 - 10 ^ ((Divisor - 1) / Divisor)) - PI / 2 / (1 - 10) * (1 - 10 ^ ((Divisor - 2) / Divisor))) / (ForThetaSelect1 - ForThetaSelect0) ‘if it happens to the last angle interval, special treatment needs since the interval width differs from middle points. RA is the scattering angle selected.
                    trial5 = trial5 + 1    ‘moves to the next random point. The pointer is increased by 1 
    
                    GOTO 150     ‘go back and repeat. The last pointer value (trial5) becomes flyC+1. Random number assigned for flyC+1 in the matrix RL(flyC+1) is ‘ridiculously large’ number, to make sure “GOTO 150 will not happen after flyC+1. 
                ELSE
                END IF
    
                ForThetaSelect0 = ForThetaSelect1   ‘for integration. The integrated value is saved as the base line, to be added with the increased value from the next angle interval, to be calculated in the next angle point. 
    
            NEXT ANGLE2
    
    ‘lines 615 to 622 is to go through the flying distance group, from the last one to the first one, randomly pick one before the current one and switch the value. This is used to disorder the ordered scattering angle and randomly assign it them to different free flying distances. 
            FOR trial99 = flyC TO 2 STEP -1   ‘go through all distances from the last one to the first one
                jj = INT(RND * trial99) + 1    ‘randomly pick a number smaller than trial99
                qqq = RA(trial99)  ‘save scattering angle of trial99 temporarily 
                RA(trial99) = RA(jj)   ‘transfer randomly picked scattering angle to trial99
                RA(jj) = qqq  ‘transfer temporarily save value to the randomly picked free flying distance. Hence swapping finishes
            NEXT trial99  ‘go through all random number in the group
    
    
            totalRE = 0   ‘prepare for the energy loss calculation 
    
            FOR trial6 = 1 TO flyC  ‘go through each free flying distance in the group
    
                ElecREup = ((EP(rum) + 511) * (SIN(RA(trial6))) ^ 2 + MASSSUB * 931 * 1000 * (1 - COS(RA(trial6)))) * EP(rum) * (EP(rum) + 2 * 511)   ‘for calculation of energy transfer 
                ElecREdown = (EP(rum) + MASSSUB * 931.5 * 1000) ^ 2 - EP(rum) * (EP(rum) + 2 * 511) * (COS(RA(trial6))) ^ 2 ‘for calculation of energy transfer
                RE(trial6) = ElecREup / ElecREdown ‘for calculation of energy transfer
                totalRE = totalRE + RE(trial6)  ‘adding all energy loss with the group
            NEXT trial6  ‘go through the whole free flying distance group 
    
            ''''''''''''''''''''''''''''''''''''''''''''''
            Ltotal = Ltotal + Lselected    ‘adding all free flying distances to get total flying distance for the whole group 
            CALL IoniElecLoss(ZSUB, SUBDENSITY * 1E24, EP(rum) * correctFactor)  ‘calculate energy loss due to ionization 
    
            IoniEloss2 = (IoniEloss2 + ABS(IoniEloss2)) / 2.0  ‘make sure the value is positive 
    
    
            CALL BremsELoss(ZSUB, SUBDENSITY * 1E24, EP(rum) * correctFactor) ‘calculate energy loss due to braking irradiation 
            BEloss = (BEloss + ABS(BEloss)) / 2   ‘make sure the value is positive
    
            ''   w10 = IoniEloss2 + BEloss 'total non-Mott scattering energy loss
            ''  w11 = EP(rum) 'electron energy 
            ''  PRINT #14, USING "##.###^^^^^ ##.###^^^^^"; w11; w10 'optional to get non-Mott energy loss as a function of energy
    
            DIM Energy1 AS DOUBLE
            DIM Energy2 AS DOUBLE
    
            Energy1 = EP(rum) - Lselected * (IoniEloss2 + BEloss) – totalRE   ‘energy of electron after both non-Mott and Mott scattering energy loss
    
    
    
    
            IF switch = 0 THEN    ‘No energy correction is needed when switch=0
            ELSE
            END IF
    
    
    
            IF switch = 1 THEN    ‘turn on energy correction is switch=1, correction below follows midpoint approximation
                IF correctFactor = 1 THEN    ‘”1” means the correction was not performed yet since “1” is the preassigned value, do the following
                    correctFactor = (Energy1 + EP(rum)) / 2 / EP(rum)  ‘the ratio of middle energy to the starting energy for the flying distance group
                    GOTO 100  ‘repeat the calculation used modified energy, utilizing correctFactor.  This means the first flying distance group will be re-peated with energy correction.  Once it is repeated,  correctFactor is not one anymore, and repeating will not happen. 
                ELSE
                END IF
    
    
                correctFactor = (Energy1 + EP(rum)) / 2 / EP(rum)   ‘calculate the correction factor of the current group, and use it for the next group
            ELSE
            END IF
    
    
            
    
    
            IF switch = 2 THEN ‘turn on the correction but the energy correction follows the implicit method
                IF correctFactor = 1 THEN ‘”1” means the correction was not performed yet since “1” is the preassigned value, do the following
                    correctFactor = Energy1 / EP(rum) ‘ratio is the final energy to the initial energy of the group. 
                    GOTO 100 ‘repeating the calculation for the first group. 
                ELSE
                END IF
                correctFactor = Energy1 / EP(rum) ‘calculate the correction factor of the current group, and use it for the next group
    
            ELSE
            END IF
    
            EP(rum) = Energy1   ‘assign the final energy of the group as an updated energy as the starting energy of the next group 
    
    
    
            '  PRINT #10, USING "##.###^^^^^"; correctFactor ‘optional for saving correctFactor
    
    
            E0 = (XP(rum) ^ 2 + YP(rum) ^ 2) ^ 0.5    ‘lateral distance from z axis
            E1 = ZP(rum)          ‘depth                           
            E2 = Lselected * IoniEloss2  ‘ionization energy loss rate
            E3 = totalRE ‘Mott scattering energy loss
    
    
            IF E0 < SUBWINDOW AND E1 > 0 AND E1 < SUBWINDOW THEN  ‘for point with the valid region between two boundaries
                e_output(INT(1 + E0 / e_deltaLat), INT(1 + E1 / e_deltaDep), 2) = e_output(INT(1 + E0 / e_deltaLat), INT(1 + E1 / e_deltaDep), 2) + E2     ‘saving ionization energy to 2-D position matrix
                e_output(INT(1 + E0 / e_deltaLat), INT(1 + E1 / e_deltaDep), 3) = e_output(INT(1 + E0 / e_deltaLat), INT(1 + E1 / e_deltaDep), 3) + E3 ‘saving Mott scattering energy loss to 2-D position matrix
            ELSE
            END IF
    
            '     PRINT #7, USING "##.###^^^^^ ##.###^^^^^ ##.###^^^^^ ##.###^^^^^"; E0; E1; E2; E3 ‘optional
    
    
            FOR trial9 = 1 TO flyC  ‘for each collision after each flying distance, calculate the direction 
                ZP(rum + trial9) = ZP(rum + trial9 - 1) + RN(trial9) * 10 ^ 8 * COS(THETAP(rum + trial9 - 1))
                XP(rum + trial9) = XP(rum + trial9 - 1) + RN(trial9) * 10 ^ 8 * SIN(THETAP(rum + trial9 - 1)) * COS(ALPHAP(rum + trial9 - 1))
                YP(rum + trial9) = YP(rum + trial9 - 1) + RN(trial9) * 10 ^ 8 * SIN(THETAP(rum + trial9 - 1)) * SIN(ALPHAP(rum + trial9 - 1))
    
    
    
    ‘calculate scattering angles after each collision within the free flying distance group
                THETA1RELATIVE = RA(trial9)  ‘scattering angle from each Mott scattering with respect to electron flying direction prior to collision 
                THETA2RELATIVE = (PI - RA(trial9)) / 2 ‘approximation for target atom
                CALL AMAGIC(THETAP(rum + trial9 - 1), ALPHAP(rum + trial9 - 1), THETA1RELATIVE, THETA2RELATIVE)  ‘convert to angles with respect to xyz coordinate
    
    ‘assign scattering angles to projectile and target atoms for each collision
                THETAP(rum + trial9) = THETA1   ‘for electron
                ALPHAP(rum + trial9) = ALPHA1  ‘for electron
                THETA2(rum + trial9) = THETA2 ‘for target atom
                ALPHA2(rum + trial9) = ALPHA2 ‘for target atom
    
            NEXT trial9    ‘finish all distance within the group
    
    
    ‘print on the screen, number of electrons simulated out of the total to be calculated 
            LOCATE 4, 47 'modified 11/21/23
            PRINT " SIMULATION:"; SIMUL; "("; SIMULS; ")" 'modified 01/12/01
            
    
    
            IF Ptype(rum) = 1 THEN LINE (INT(ZP(rum) / SUBWINDOW * WINDOWX), INT((0.5 + XP(rum) / SUBWINDOW) * WINDOWY))-(INT(ZP(rum + flyC) / SUBWINDOW * WINDOWX), INT((0.5 + XP(rum + flyC) / SUBWINDOW) * WINDOWY)), 0 'plot a line connecting the current and next electron position
            '    IF Ptype(rum) = 0 THEN LINE (INT(ZP(rum) / SUBWINDOW * WINDOWX), INT((0.5 + XP(rum) / SUBWINDOW) * WINDOWY))-(INT(ZP(rum + flyC) / SUBWINDOW * WINDOWX), INT((0.5 + XP(rum + flyC) / SUBWINDOW) * WINDOWY)), 1 'Option to plot for target atom
    
    
            NewDeltaZ = ZP(rum + flyC) - ZP(rum)
            NewDeltaX = XP(rum + flyC) - XP(rum)
            NewDeltaY = YP(rum + flyC) - YP(rum)
            NewTheta = THETAP(rum + flyC)
            NewAlpha = ALPHAP(rum + flyC)
            NewEP = EP(rum)
    
            '   IF NewTheta > PI / 2 AND ZP(rum + flyC) < 0 THEN PRINT " Theta="; NewTheta; "ZP(rum)="; ZP(rum + flyC)
    
            '    IF NewTheta > PI / 2 AND ZP(rum + flyC) < 0 THEN
            '   PRINT "NewTheta="; NewTheta; "ZP(rum+flyC)="; ZP(rum + flyC)
            '    INPUT xxx
            '  ELSE
            ' END IF
    
    
            '''''''''    CALL IMAGE(Scan(), NewDeltaZ, NewDeltaX, NewDeltaY, NewTheta, NewAlpha, NewEP, Resolution, sizeScan)
    
            '    INPUT bnb
            '      SUB IMAGE (Scan(),DeltaZima, DeltaXima, DeltaYima, Thetaima, Alphaima, Eima, Resolution, size)
    
    
    ‘lines 797 to 805, update position after each flying distance group, transfer other information needed.  
            ZP(rum) = ZP(rum + flyC)
            XP(rum) = XP(rum + flyC)
            YP(rum) = YP(rum + flyC)
            THETAP(rum) = THETAP(rum + flyC)
            ALPHAP(rum) = ALPHAP(rum + flyC)
            CP(rum) = CP(rum) 'CP(rum + 1) = CP(rum)
            mp(rum) = mp(rum) 'mp(RUM + 1) = mp(RUM)
            EP(rum) = EP(rum) '- ElecRE - Lselected * (SUBDENSITY * 1E24) * TotalelasticE
            Ptype(rum) = Ptype(rum)
    
    
    
            mmmm = 1
            FOR trial8 = 1 TO flyC
    
                IF RE(trial8) >= ion_Ed THEN
                    ZP(rum + mmmm) = ZP(rum + trial8)
                    XP(rum + mmmm) = XP(rum + trial8)
                    YP(rum + mmmm) = YP(rum + trial8)
                    CP(rum + mmmm) = Z2
                    mp(rum + mmmm) = MASS2
                    THETAP(rum + mmmm) = THETA2(rum + trial8)
                    ALPHAP(rum + mmmm) = ALPHA2(rum + trial8)
                    Ptype(rum + mmmm) = 0
                    EP(rum + mmmm) = RE(trial8)
    
    
    
                    E0 = (XP(rum + mmmm) ^ 2 + YP(rum + mmmm) ^ 2) ^ 0.5
                    E1 = ZP(rum + mmmm)
                    IF E0 < SUBWINDOW AND E1 > 0 AND E1 < SUBWINDOW THEN
                        e_output(INT(1 + E0 / e_deltaLat), INT(1 + E1 / e_deltaDep), 4) = e_output(INT(1 + E0 / e_deltaLat), INT(1 + E1 / e_deltaDep), 4) + 1
                        'the above is to record one vacancy created by electron
                        E4V = EP(rum)
                        PRINT #13, USING "##.###^^^^^ ##.###^^^^^"; E4V; 1
                    ELSE
                    END IF
                    '       INPUT XXX
                    mmmm = mmmm + 1
                ELSE
                END IF
    
            NEXT trial8
    
    
    
    
    
    
    
            rum = rum + (mmmm - 1)
    
    
    
            '  B1 = Ltotal
            '   B2 = EP(rum)
            '     B3 = THETAP(rum)
            '     B4 = Lselected
    
            '     PRINT #10, USING "##.###########^^^^^ ##.###^^^^^ ##.###^^^^^"; B2; B4; B4 / B2
    
    
            '      IF THETAP(rum) = 0 THEN PRINT "something is wrong here and check thetap, why =0"
            GOTO 100
    
        ELSE
        END IF
        '  ELSE
        ' END IF
    
    
    
        'end of electron irradiation
        600
    NEXT SIMUL
    
    
    '''''''FOR m = 0 TO sizeScan
    ''''''FOR mm = 0 TO sizeScan
    '''''''PRINT #12, USING "##.###^^^^^ ##.###^^^^^ ##.###^^^^^"; scandelta * (m - sizeScan / 2); scandelta * (mm - sizeScan / 2); Scan(m, mm, 8)
    '''''NEXT mm
    ''''''''NEXT m
    
    
    
    
    
    
    
    
    '''''FOR RRR = 0 TO SUBWINDOW
    '''''PRINT #1, USING "##.###^^^^^ ##.###^^^^^"; RRR; PR(RRR)
    '''''NEXT RRR
    ''''FOR RRRR = 0 TO SUBWINDOW
    '''''PRINT #2, USING "##.###^^^^^ ##.###^^^^^"; RRRR; I(RRRR)
    '''''PRINT #3, USING "##.###^^^^^ ##.###^^^^^"; RRRR; V(RRRR)
    
    '''''NEXT RRRR
    
    FOR N0 = 1 TO e_interval
        FOR N1 = 1 TO e_interval
            FOR N2 = 1 TO 4
                Backup_e_output(N0, N1, N2) = e_output(N0, N1, N2)
            NEXT N2
        NEXT N1
    NEXT N0
    
    
    FOR R0 = 1 + e_ave_range TO e_interval - e_ave_range
        FOR R1 = 1 + e_ave_range TO e_interval - e_ave_range
            combine1 = 0
            combine2 = 0
            combine3 = 0
            combine4 = 0
    
            FOR R2 = -e_ave_range TO e_ave_range
                FOR R22 = -e_ave_range TO e_ave_range
    
                    combine1 = Backup_e_output(R0 + R2, R1 + R22, 1) + combine1
                    combine2 = Backup_e_output(R0 + R2, R1 + R22, 2) + combine2
                    combine3 = Backup_e_output(R0 + R2, R1 + R22, 3) + combine3
                    combine4 = Backup_e_output(R0 + R2, R1 + R22, 4) + combine4
                NEXT R22
            NEXT R2
            e_output(R0, R1, 1) = combine1 / (2 * e_ave_range + 1) ^ 2
            e_output(R0, R1, 2) = combine2 / (2 * e_ave_range + 1) ^ 2
            e_output(R0, R1, 3) = combine3 / (2 * e_ave_range + 1) ^ 2
            e_output(R0, R1, 4) = combine4 / (2 * e_ave_range + 1) ^ 2
        NEXT R1
    NEXT R0
    
    
    FOR W0 = 1 TO e_interval
        accum = 0
        FOR w1 = 1 TO e_interval
            accum = accum + e_output(w1, W0, 1)
        NEXT w1
        PRINT #9, USING "##.###^^^^^ ##.###^^^^^"; (W0 - 1 + 0.5) * e_deltaDep / 10000; accum / SIMULS / ((e_deltaDep / 1E4) * 1E4 * 1E4)
    NEXT W0
    
    
    test_e = 0
    test_IoniE = 0
    test_RecE = 0
    test_Disp = 0
    
    FOR R3 = 1 TO e_interval
        FOR R4 = 1 TO e_interval
            unit_volume = 2 * PI * (R3 + 0.5) * e_deltaLat * e_deltaLat * e_deltaDep / 10 ^ 12 'unit is micron^3
            PRINT #7, USING "##.###^^^^^ ##.###^^^^^ ##.###^^^^^ ##.###^^^^^ ##.###^^^^^ ##.###^^^^^"; (R4 - 1 + 0.5) * e_deltaDep / 10000; (R3 - 1 + 0.5) * e_deltaLat / 10000; e_output(R3, R4, 1) / unit_volume / SIMULS; e_output(R3, R4, 2) / unit_volume / SIMULS; e_output(R3, R4, 3) / unit_volume / SIMULS; e_output(R3, R4, 4) / unit_volume / SIMULS
            test_e = test_e + unit_volume * e_output(R3, R4, 1) / unit_volume / SIMULS
            test_IoniE = test_IoniE + e_output(R3, R4, 2) / unit_volume / SIMULS * unit_volume
            test_RecE = test_RecE + e_output(R3, R4, 3) / unit_volume / SIMULS * unit_volume
            test_Disp = test_Disp + e_output(R3, R4, 4) / unit_volume / SIMULS * unit_volume
        NEXT R4
    NEXT R3
    PRINT "integrated total electron inside ="; test_e
    PRINT "integrtated total ioniztion (inelastic) energy loss="; test_IoniE
    PRINT "integrated total elastic energy loss="; test_RecE
    PRINT "integrated total displacement creation="; test_Disp
    END
    
    SUB AMAGIC (THETAO, ALPHAO, THETA1RELATIVE, THETA2RELATIVE)
        SHARED PI
        SHARED THETA1 'RESULT TO MAIN PROGRAM, DEFLECTION ANGLE TO ORIGINAL C.
        SHARED ALPHA1 'RESULT TO MAIN PROGRAM
        SHARED THETA2 'RESULT TO MAIN PROGRAM
        SHARED ALPHA2 'RESULT TO MAIN PROGRAM
        ALPHA1RELATIVE = RND * 2 * PI
        ALPHA2RELATIVE = ALPHA1RELATIVE + PI
        '=====FOR PRIMARY ION
        X1 = SIN(THETA1RELATIVE) * COS(ALPHA1RELATIVE)
        Y1 = SIN(THETA1RELATIVE) * SIN(ALPHA1RELATIVE)
        Z1 = COS(THETA1RELATIVE)
    
        Y0 = Y1 * COS(THETAO) + Z1 * SIN(THETAO)
        Z0 = -Y1 * SIN(THETAO) + Z1 * COS(THETAO)
        X0 = X1
    
        Z = Z0
        X = X0 * SIN(ALPHAO) + Y0 * COS(ALPHAO)
        Y = -X0 * COS(ALPHAO) + Y0 * SIN(ALPHAO)
    
        IF Z > O THEN THETA1 = ATN(SQR(X ^ 2 + Y ^ 2) / Z)
        IF Z = 0 THEN THETA1 = PI / 2
        IF Z < 0 THEN THETA1 = PI + ATN(SQR(X ^ 2 + Y ^ 2) / Z)
    
        IF SIN(THETA1) <> 0 THEN
            IF X > 0 THEN ALPHA1 = ATN(Y / X)
            IF X = 0 THEN ALPHA1 = PI - SGN(Y) * PI / 2
            IF X < 0 THEN ALPHA1 = PI + ATN(Y / X)
        ELSE
            ALPHA1 = 0
        END IF
    
        '========FOR RECOIL TARGET ION
    
        X1 = SIN(THETA2RELATIVE) * COS(ALPHA2RELATIVE)
        Y1 = SIN(THETA2RELATIVE) * SIN(ALPHA2RELATIVE)
        Z1 = COS(THETA2RELATIVE)
    
        Y0 = Y1 * COS(THETAO) + Z1 * SIN(THETAO)
        Z0 = -Y1 * SIN(THETAO) + Z1 * COS(THETAO)
        X0 = X1
    
        Z = Z0
        X = X0 * SIN(ALPHAO) + Y0 * COS(ALPHAO)
        Y = -X0 * COS(ALPHAO) + Y0 * SIN(ALPHAO)
    
        IF Z > O THEN THETA2 = ATN(SQR(X ^ 2 + Y ^ 2) / Z)
        IF Z = 0 THEN THETA2 = PI / 2
        IF Z < 0 THEN THETA2 = PI + ATN(SQR(X ^ 2 + Y ^ 2) / Z)
    
        IF SIN(THETA2) <> 0 THEN
            IF X > 0 THEN ALPHA2 = ATN(Y / X)
            IF X = 0 THEN ALPHA2 = PI - SGN(Y) * PI / 2
            IF X < 0 THEN ALPHA2 = PI + ATN(Y / X)
        ELSE
            ALPHA2 = 0
        END IF
    
    
    END SUB
    
    FUNCTION DF (X, COLUMBIAVK, Z1, Z2, AU)
        DF = F(X, COLUMBIAVK, Z1, Z2, AU) / X + COLUMBIAVK / X * (.35 * EXP(-.3 / X / AU) * .3 / AU + .55 * EXP(-1.2 / X / AU) * 1.2 / AU + .1 * EXP(-6 / X / AU) * 6 / AU)
    END FUNCTION
    
    SUB EMAGIC (X)
        SHARED ATOMD
        SHARED ELOSS
    
        ' X IS THE CURVE FITTING FROM TRIM AND UNIT IS EV, X UNIT IS KEV
        IF X * 1000 > 1 AND X * 1000 <= 10 THEN ELOSS = ATOMD * (.7122 + .1026 * X * 1000) / 1000
        IF X * 1000 > 10 AND X * 1000 <= 100 THEN ELOSS = ATOMD * (1.671 + .0252 * X * 1000) / 1000
        IF X * 1000 > 100 AND X * 1000 <= 400 THEN ELOSS = ATOMD * (5.0767 + .0044 * X * 1000) / 1000
        IF X * 1000 > 1000 AND X * 1000 <= 10000 THEN ELOSS = ATOMD * (9.28 + .00144 * X * 1000) / 1000
        IF X * 1000 > 10000 THEN ELOSS = ATOMD * (9.28 + .00144 * X * 1000) / 1000
    END SUB
    
    FUNCTION F (X, COLUMBIAVK, Z1, Z2, AU)
        '===========================UNIVERAIAL SCREENING POTENTIAL=================
        IF X = O THEN
            F = 0
        ELSE
            F = COLUMBIAVK * X * (.35 * EXP(-.3 / X / AU) + .55 * EXP(-1.2 / X / AU) + .1 * EXP(-6 / X / AU))
        END IF
        '=========================COLUMBIA POTENTIAL==============================
        'F = COLUMBIAVK * X
    
    
    END FUNCTION
    
    SUB TMAGIC (MASS1, Z1, MASS2, Z2, INELAB, P)
    
        SHARED ATOMD 'WHAT SHARED IS ALL THE NEEDED DATA BUT NOT DEFINED HERE
        'AND ALL THE RESULT NEEDING TRANSFERED TO MAIN PROGRAM
        SHARED THETA1RELATIVE 'RESULT TO MAIN PROGRAM
        SHARED THETA2RELATIVE 'RESULT TO MAIN PROGRAM
        SHARED RE 'RESULT TO MAIN PROGRAM
    
        PI = 3.1415926#
        '=============================INITIAL PARAMETER============================
    
        COLUMBIAVK = .0143992# * Z1 * Z2 'POTENTIAL V=COLUMBIAVK(ANSTRON*KEV)*Z1*Z2/R '
        MC = MASS1 * MASS2 / (MASS1 + MASS2) 'MC--REDUCED MASS IN CENTER-MASS COORDINATOR
        INVLAB = SQR(INELAB * 2 / MASS1) 'INVLAB--INCIDENT VOLOCITY IN LAB C.
        EC = 1 / 2 * MC * INVLAB ^ 2 'EC--INITIAL ENERGY IN CM
        AU = .8854 * .529 / (Z1 ^ .5 + Z2 ^ .5) ^ (2 / 3)
        ELINHARD = EC * AU / COLUMBIAVK
    
        '==============FIND RMIN FOR DIFFERENT ENERGY======================
        AA = P ^ 2
        IF AA = 0 THEN AA = .00001
        BB = COLUMBIAVK / EC
        CC = -1
        COLUMRMIN = 1 / 2 / AA * (-BB + SQR(BB ^ 2 - 4 * AA * CC)) 'COLUMRMIN IS 1/RMIN, RMIN IS THE MINIMUM R UNDER COLUMBIA POTENTIAL
    
        CALTIME = 1
        300 RMINTRY1 = COLUMRMIN
        DV = ABS(-DF(RMINTRY1, COLUMBIAVK, Z1, Z2, AU) / EC - 2 * P ^ 2 * RMINTRY1)
        IF ABS(DV) < .000001 THEN DV = .1
        RMINTRY2 = RMINTRY1 + (1 - F(RMINTRY1, COLUMBIAVK, Z1, Z2, AU) / EC - P ^ 2 * RMINTRY1 ^ 2) / DV
        COLUMRMIN = RMINTRY2
        CALTIME = CALTIME + 1
        IF CALTIME > 10000 THEN GOTO 350
        IF ABS(RMINTRY2 - RMINTRY1) > .00001 THEN GOTO 300
        350 RMIN = (RMINTRY2 + RMINTRY1) / 2
    
    
    
    
        '=====================CALCULATE DEFELCTION ANGLE=========================
    
    
    
        RBIERSACK = 2 * (EC - F(RMIN, COLUMBIAVK, Z1, Z2, AU)) / RMIN ^ 2 / DF(RMIN, COLUMBIAVK, Z1, Z2, AU)
    
        BBIERSACK = P / AU
        ROBIERSACK = 1 / (RMIN * AU)
        RCBIERSACK = RBIERSACK / AU
    
        C1BIERSACK = .6743
        C2BIERSACK = .009611
        C3BIERSACK = .005175
        C4BIERSACK = 10!
        C5BIERSACK = 6.314
        ALTHABIERSACK = 1 + C1BIERSACK * ELINHARD ^ (-1 / 2)
        BELTABIERSACK = (C2BIERSACK + ELINHARD ^ (1 / 2)) / (C3BIERSACK + ELINHARD ^ (1 / 2))
        GAMABIERSACK = (C4BIERSACK + ELINHARD) / (C5BIERSACK + ELINHARD)
        ABIERSACK = 2 * ALTHABIERSACK * ELINHARD * BBIERSACK ^ BELTABIERSACK
        GBIERSACK = GAMABIERSACK * 1 / ((1 + ABIERSACK ^ 2) ^ (1 / 2) - ABIERSACK)
        DELTABIERSACK = ABIERSACK * (ROBIERSACK - BBIERSACK) / (1 + GBIERSACK)
        'PRINT "COS(H/2)="; (BBIERSACK + RCBIERSACK + DELTABIERSACK) / (ROBIERSACK + RCBIERSACK)
        'END
        IF P = 0 THEN
            CALPHA1 = PI
        ELSE
            CALPHA1 = 2 * ATN(SQR((ROBIERSACK + RCBIERSACK) ^ 2 / (BBIERSACK + RCBIERSACK + DELTABIERSACK) ^ 2 - 1))
        END IF
        ' ===================DEFLECTION ANGLE IN LAB COORDINATOR===========
        COSPLUSMASS = COS(CALPHA1) + MASS1 / MASS2
        IF COSPLUSMASS = 0 THEN THETA1RELATIVE = PI / 2
        IF COSPLUSMASS < 0 THEN THETA1RELATIVE = PI + ATN(SIN(CALPHA1) / COSPLUSMASS)
        IF COSPLUSMASS > 0 THEN THETA1RELATIVE = ATN(SIN(CALPHA1) / COSPLUSMASS)
    
        '===================CALCUTE T & DIRECTION OF TARGET ATOM=============
        RE = 4 * EC * MC / MASS2 * SIN(CALPHA1 / 2) * SIN(CALPHA1 / 2)
        'RE IS ENERGY TRANSIMITED TO TARGET ATOM'
        '===================RECOILED DIRECTION==================================
        THETA2RELATIVE = (PI - CALPHA1) / 2 'RECOILED DIRECTION
        'PRINT "RE="; RE, "CALPHA1="; CALPHA1
    END SUB
    
    SUB DSCMOTT (mott(), scr(), ElectronEnergy, SubstrateZ, Theta)
    
    
        SHARED DiffCross AS DOUBLE
    
        PI = 3.1415926#
        beta = (1 - (ElectronEnergy / 511 + 1) ^ (-2)) ^ 0.5
        betaAVE = 0.7181287
        gamma = ElectronEnergy / 511 + 1
        elecmass = 1 'specific for a.u. unit,  only for Feq calcu.
        Vc = 137 'speed of light for a.u. unit, only for Feq calcu.
        re1 = 2.817938E-13 'in the unit of cm)
    
        Rmott = 0
        FOR xx = 1 TO 5
            alpha1 = 0
            FOR yy = 1 TO 6
                alpha1 = alpha1 + mott(SubstrateZ, xx, yy) * (beta - betaAVE) ^ (yy - 1)
            NEXT yy
            Rmott = Rmott + alpha1 * (1 - COS(Theta)) ^ ((xx - 1) / 2)
        NEXT xx
        DSC = (SubstrateZ * re1) ^ 2 * (1 - beta ^ 2) / beta ^ 4 * (1 - COS(Theta)) ^ (-2) 'Differential Ruth cross section
        moment = 2 * beta * gamma * elecmass * Vc * SIN(Theta / 2)
        Feq = 0
        FOR ww = 1 TO 3
            Feq = Feq + scr(SubstrateZ, ww) * scr(SubstrateZ, ww + 3) ^ 2 / (scr(SubstrateZ, ww + 3) ^ 2 + moment ^ 2)
        NEXT ww
        DiffCross = Rmott * DSC * (1 - Feq) ^ 2
    
        'PRINT "Rmott="; Rmott
        'PRINT "DSC="; DSC
        'PRINT "Feq="; Feq
        'PRINT "moment="; moment
        'PRINT "SubstrateZ="; SubstrateZ
        'PRINT "scr(SubstrateZ,1)="; scr(SubstrateZ, 1)
        '  PRINT "DiffCross="; DiffCross
    
        'INPUT XXXXXX
    
    END SUB
    
    SUB IoniElecLoss (ZSUB, DENSITY, ElecE)
        ' SHARED IoniEloss2 AS DOUBLE
        '  SHARED IoniEloss1 AS DOUBLE
    
        ''  PRINT "ElecE="; ElecE
    
        beta = (1 - (ElecE / 511 + 1) ^ (-2)) ^ 0.5
        KforLoss = 0.734 * ZSUB ^ 0.037
        IF ZSUB < 13 THEN JforLoss = 0.0115 * ZSUB
        IF ZSUB >= 13 THEN JforLoss = 0.00976 * ZSUB + 0.0585 * ZSUB ^ (-0.19)
    
        JforLoss = JforLoss / (1 + KforLoss * JforLoss / ElecE)
    
        IoniEloss2 = 153.55 * DENSITY * ZSUB / (6.022 * 10 ^ 23) / beta ^ 2 * (LOG(511 * beta ^ 2 * ElecE / (2 * JforLoss ^ 2 * (1 - beta ^ 2))) - LOG(2) * (2 * (1 - beta ^ 2) ^ 0.5 - 1 + beta ^ 2) + 1 - beta ^ 2 + 1 / 8 * (1 - (1 - beta ^ 2) ^ 0.5) ^ 2)
    
        '  JforLoss = JforLoss / (1 + KforLoss * JforLoss / ElecE)
        '  IoniEloss1 = 7.85 * (10 ^ 4) * DENSITY * ZSUB / (6.022 * 10 ^ 23) / ElecE * LOG(1.166 * ElecE / JforLoss)
        '  IF IoniEloss2 < IoniEloss1 THEN IoniEloss2 = IoniEloss1
    
    END SUB
    
    SUB BremsELoss (ZSUB, DENSITY, ElecE)
        '   SHARED BEloss AS DOUBLE
        BEloss = 1.4 * 10 ^ (-4) * DENSITY / (6.022 * 10 ^ 23) * ZSUB * (ZSUB + 1) * (ElecE + 511) * (4 * LOG(2 * (ElecE + 511) / 511) - 4 / 3)
        'PRINT "what the help density="; DENSITY
        ' PRINT "what the help ElecE="; ElecE
        '    PRINT "BEloss inside="; BEloss
    END SUB
    
    
    SUB IMAGE (Scan(), DeltaZima, DeltaXima, DeltaYima, Thetaima, Alphaima, Eima, Resolution, size)
        '  Scan(sizeScan + 1, sizeScan + 1, 6)
        SHARED Scan()
        '  PRINT "ScanZ="; Scan(20, 20, 3)
        ' INPUT xx
    
        PI = 3.14159
        m = 1
    
        FOR i = 0 TO size
            FOR j = 0 TO size
                FOR k = 1 TO Resolution
                    Scan(i, j, 1) = Scan(i, j, 1) + DeltaXima / Resolution
                    Scan(i, j, 2) = Scan(i, j, 2) + DeltaYima / Resolution
                    Scan(i, j, 3) = Scan(i, j, 3) + DeltaZima / Resolution
                    999
    
                    IF (Scan(i, j, 1) ^ 2 + Scan(i, j, 2) ^ 2 + (Scan(i, j, 3) - 200) ^ 2) ^ 0.5 < 50 THEN
    
                        Scan(i, j, 1) = Scan(i, j, 1) + DeltaXima / Resolution
                        Scan(i, j, 2) = Scan(i, j, 2) + DeltaYima / Resolution
                        Scan(i, j, 3) = Scan(i, j, 3) + DeltaZima / Resolution
    
    
                        GOTO 999
    
                    ELSE
                    END IF
                NEXT k
                Scan(i, j, 4) = Thetaima
                Scan(i, j, 5) = Alphaima
                Scan(i, j, 6) = Eima
    
                '  PRINT "Scan(Z)="; Scan(i, j, 3); "DeltaZima="; DeltaZima; "Thetaima="; Thetaima; "Resolution="; Resolution; "size="; size
                ' INPUT xx
    
    
                '  IF Scan(i, j, 3) < 0 AND Scan(i, j, 7) = 0 AND Scan(i, j, 4) > 150 * PI / 180 AND Scan(i, j, 4) < 170 * PI / 180 THEN
                ' PRINT #12, USING "##.###^^^^^ ##.###^^^^^ ##.###^^^^^ ##.###^^^^^ ##.###^^^^^"; Scan(i, j, 1); Scan(i, j, 2); Scan(i, j, 4); Scan(i, j, 5); Scan(i, j, 6)
                ' Scan(i, j, 7) = 1
                '        PRINT "something occurs here"
                'ELSE
                'END IF
                '  INPUT bb
    
    
                IF Scan(i, j, 3) < 0 AND Scan(i, j, 7) = 0 THEN
                    Scan(i, j, 8) = Scan(i, j, 8) + 1
                    Scan(i, j, 7) = 1
                ELSE
                END IF
    
                ' PRINT #12, USING "##.###^^^^^ ##.###^^^^^ ##.###^^^^^ ##.###^^^^^ ##.###^^^^^"; Scan(i, j, 1); Scan(i, j, 2); Scan(i, j, 4); Scan(i, j, 5); Scan(i, j, 6)
                ' Scan(i, j, 7) = 1
                '        PRINT "something occurs here"
                'ELSE
                'END IF
    
    
    
            NEXT j
        NEXT i
        '    PRINT "Scan(20,20,3)="; Scan(20, 20, 3); "Scan(20,20,4)="; Scan(20, 20, 4)
    END SUB
    
    
    '       Scan(i, j, 1) = e_deltaLat * (i - sizeScan / 2) 'scanning
    '      Scan(i, j, 2) = e_deltaLat * (i - sizeScan / 2) 'scanning
    '     Scan(i, j, 3) = 0 'depth scanning
    '    Scan(i, j, 4) = 0 'theta scanning
    '   Scan(i, j, 5) = 0 'alpha scanning
    '  Scan(i, j, 6) = ElecE0 'E scanning
    
    ' FOR i = 0 TO sizeScan 'scanning
    '    FOR j = 0 TO sizeScan 'scanning
\end{minted}

\end{document}